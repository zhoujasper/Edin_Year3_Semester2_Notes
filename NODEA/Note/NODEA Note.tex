\documentclass[9pt]{article}

\usepackage{graphicx} % Required for inserting images
\usepackage{amsmath}
\usepackage{amsfonts}
\usepackage{ctex}
\usepackage{enumitem}
\usepackage{longtable}
\usepackage{makecell} % 换行

% 使用分栏宏包
\usepackage{multicol} 
\usepackage{multirow}
\setlength{\columnseprule}{0.4pt} % 分割线

% 设置字体
\usepackage{unicode-math}
\setmainfont{Cambria}
\setmathfont{Cambria Math}

% 调整页面布局
\usepackage[a4paper, top=0.7cm, bottom=1cm, left=0.7cm, right=.7cm]{geometry}
\setlength{\footskip}{15pt}

% 设置页脚/页眉
\usepackage{fancyhdr}
\fancyfoot[C]{Copyright By Jingren Zhou | Page \thepage}
\fancyhead[]{}
\pagestyle{fancy}
% 去除线
\renewcommand{\headrulewidth}{0pt}
\renewcommand{\footrulewidth}{0pt}

% 设置 section/subsection 之间的行间距
\usepackage{titlesec}
\titlespacing*{\section}{0pt}{0pt}{0pt}
\titlespacing*{\subsection}{0pt}{0pt}{0pt}

% 调整标题上下间距
\usepackage{titling}
\setlength{\droptitle}{-2.4cm} % 负值表示向上移动

% 设置标题,作者,时间
\title{NODEA Note}
\author{}
\date{}

% 正文
\begin{document}


% 标题
\maketitle
\thispagestyle{fancy}
\vspace{-3.5cm}

% 字体大小
\fontsize{10pt}{11pt}\selectfont
\setlength{\parindent}{8pt}


\section{Basic Knowledge} % Basic Knowledge

\textbf{Def of ODE \& ODEs}: { (1st order)} ODE: $\frac{dy}{dt}=f(t,y)$ \quad \& \quad ODEs: $\frac{d\mathbf{y}}{dt}=\mathbf{f}(t,\mathbf{y})$ , {\footnotesize $\mathbf{y}=(y_1,...,y_d)^T,\mathbf{f}(t,\mathbf{y})=(f_1(t,\mathbf{y}),...,f_d(t,\mathbf{y}))^T$}

\textbf{Autonomous}: $\frac{d\mathbf{y}}{dt}=\mathbf{f}(\mathbf{y})$ \ $\Rightarrow$ \ autonomous ODE(s). \quad \quad \quad || $\Downarrow$ New Autonomous ODEs: $\frac{d\mathbf{y}}{ds}=\mathbf{f}(y_{d+1},\mathbf{y})$ and $\frac{dy_{d+1}}{ds}=1$

$\cdot$ \textbf{Change to Autonomous}: For $\frac{d\mathbf{y}}{dt}=\mathbf{f}(t,\mathbf{y})$. Let $y_{d+1}=t$ and \textit{new} independent variable $s$ s.t. $\frac{dt}{ds}=1$ $\Uparrow$

\textbf{Linearity}: ODE: $\frac{dy}{dt}=f(t,y)$ is linearity if $f(t,y)=a(t)y+b(t)$ \quad || ODEs: {\small If each ODE is linear, then the ODEs are linear.}

\textbf{Picard's Theorem}: If $f(t,y)$ is continuous in $D:=\{(t,y):t_0\leq t\leq T,|y-y_0|<K\}$ and $\exists L>0$ {\footnotesize (Lipschitz constant)} s.t.

$\quad$ $\forall (t,u),(t,v)\in D \quad|f(t,u)-f(t,v)|\leq L|u-v|$ {\tiny (ps:Can use MVT)}. And Assume that $M_f(T-t_0)\leq K,M_f:=\max\{|f(t,u)|:(t,u)\in D\}$

$\quad$ $\Rightarrow$ \textbf{Then}, $\exists$ a unique continuously differentiable solution $y(t)$ to the IVP $\frac{dy}{dt}=f(t,y),y(t_0)=y_0$ on $t\in[t_0,T]$.

\textbf{Existence \& Uniqueness Theorem}: IVP $\frac{d\mathbf{y}}{dt}=\mathbf{f}(t,\mathbf{y}),\mathbf{y}(t_0)=\mathbf{y}_0$. If $f(t,y)$ and $\frac{\partial f}{\partial y_i}$ are continuous in a neighborhood of $(t_0,\mathbf{y}_0)$.

$\quad$ $\Rightarrow$ \textbf{Then}, $\exists I:=(t_0-\delta,t_0+\delta)$ s.t. $\exists$ a unique continuously differentiable solution $\mathbf{y}(t)$ to the IVP on $t\in I$.


\section{Acknowledge} % Acknowledge

\vspace{-10pt}
\begin{longtable}{|c|l||c|l|}
    \hline
    Notation & Meaning & Notation & Meaning \\
    \hline
    \hline
    $[a,b]$ & Approximate function for $t\in[a,b]$ & {\footnotesize $t_0=a \ | \ t_N=b$} & Assume that $t_0=a,t_N=b$ \\
    \hline
    $N$ & number of \textbf{timesteps} \ \ {\tiny (i.e. Break up interval $[a,b]$ into $N$ equal-length sub-intervals)} & $h$ & \textbf{stepsize} \ \ { ($h=\frac{b-a}{N}$)} \\
    \hline
    $t_i$ & Define $N+1$ points: $t_0,t_1,...,t_N$ & $t_m$ & $t_m=a+h\cdot m=t_0+h\cdot m$ \\
    \hline
    $y_i$ & Approximation of $y$ at point $t=t_i$ \ \ (Except $y_0$) & $y(t_i)$ & Exact value of $y$ at point $t=t_i$ \\
    \hline
\end{longtable}
\vspace{-10pt}


\section{Euler's Method and Taylor Series Method} % Euler's Method and Taylor Series Method

\textbf{Euler's Method Algorithm}: Approximate ODE $\frac{dy}{dt}=f(t,y),y(t_0)=y_0$ with number of steps $N$. \ (Similarly for ODEs)

\hspace{20pt} $\Rightarrow$
\textbf{for} $n=0,1,2,...,N-1$:
\quad $y_{n+1}=y_n+hf(t_0+nh,y_n)=y_n+hf(t_n,y_n)$ \quad 
\textbf{end} \hspace{80pt} { (ps: $\Downarrow$ Can get $|y''|<M$)}

\textbf{Boundedness Theorem}: {\small For $\frac{dy}{dt}=f(t,y),y(a)=y_0$ and suppose there exists a unique, twice differentiable, solution $y(t)$ on $[a,b]$.}

\quad \quad \quad {\small Suppose: $y$ is continuous and $|\frac{\partial f}{\partial y}|\leq L$. \ $\Rightarrow$ \ the solution $y_n$ given Euler's method satisfies: $e_n=|y_n-y(t_n)|\leq Dh,D=e^{(b-a)L}\frac{M}{2L}$}

$\cdot$ \textbf{Lemma}: If $v_{n+1}\leq Av_n+B$, then $v_n\leq A^nv_0+\frac{A^n-1}{A-1}B$ \quad If $v_{n}=e_n:=y_n-y(t_n)$, then $A=1+hL,B=h^2M/2$ { (suppose $|y''|<M$)}

\textbf{Order Notation ($\mathcal{O}$)}: we write $z(h)=\mathcal{O}(h^p)$ if $\exists C,h_0>0$ s.t. $|z|\leq Ch^p,0<h<h_0$

\textbf{Flow Map ($\Phi,\Psi$)}: $\Phi_{t_0,h}(y_0)=y(t_0+h)$ \quad Clearly, $\Phi(t_n+h)=y(t_n+h)=\Phi_h(y(t_n))=y(t_{n+1})$.

$\cdot$ $\Psi_{t_n,h}(y_n)=y_{n+1}$:= Numerical method for ODE \quad Clearly, $\Psi(t_n+h)=y_{n+1}=\Psi_h(y_{n})$

\textbf{Taylor Series Method}: {\small Approximate ODE $\frac{dy}{dt}=f(t,y),y(t_0)=y_0$ with \textit{n-order Methods}: 用 Taylor Series 在 $t_0+h$ 处展开 保留到 $n$ 阶}

$\cdot$ $\Phi_{t,h}(y)=y+hf(t,y)+\frac{1}{2}h^2[f_t(t,y)+f_y(t,y)f(t,y)]+\frac{1}{6}y'''(t,y)h^3+\cdots$ \qquad (For one variable $y$)

$\cdot$ ps: {\footnotesize Taylor Series: $y(t_0+h)=y(t_0)+hy'(t_0)+\frac{h^2}{2}y''(t_0)+\cdots+\frac{h^{n-1}}{(n-1)!}y^{(n-1)}(t_0)+\frac{h^n}{n!}y^{(n)}(t^*),t^*\in[t,t+h]$ \quad \quad  ps: $y'=f,y''=f_t+f_yf$}


\section{Convergence of One-Step Methods { consider for autonomous $y'=f(y)$}} % Convergence of One-Step Methods

\subsection{Convergence | Consistent | Stable} % Convergence | Consistent | Stable
\textbf{Global Error}: global error after $n$ steps: $e_n:=y_n-y(t_n)$ \quad \textbf{Local Error}: {\small For \textit{one-step} method is: $le(y,h)=\Psi_h(y)-\Phi_h(y)$}

\textbf{Consistent}: {\small If $||le(y,h)||\leq Ch^{p+1} (\leq \mathcal{O}(h^{p+1})), \ C>0$. $\Rightarrow$ Consistent at order $p$.} \quad \textbf{Stable}: If $||\Psi_h(u)-\Psi_h(v)||\leq(1+h\widehat{L})||u-v||$

\textbf{Convergent}: {\small A method is convergent if:$\forall \ T$, $\lim\limits_{h\to0,\ h=T/N}\max\limits_{n=0,1,...,N}||e_n||=0$} \quad \quad \quad {\small $\Downarrow$ Then the global error satisfies: $\max\limits_{n=0,1,...,N}||e_n||=\mathcal{O}(h^p)$} \ {\tiny \textbf{p-th order}}

\vspace{-2pt}
\textbf{Convergence of One-Step Method}: {\small For $y'=f(y)$, and a one-step method $\Psi_h(y)$ is $^1$ consistent at order $p$ and $^2$ stable with $\widehat{L}\Uparrow$.} \ {\tiny (ps:$C=\frac{C}{\widehat{L}}(e^{T\widehat{L}}-1)$)}


\subsection{More One-Step Methods | Runge-Kutta Methods | Collocation} % More One-Step Methods | Runge-Kutta Methods | Collocation

\textbf{Construction of More General one-step Method}: For $y'=f(y),y(t_0)=y_0$ \ $\Rightarrow$ \ $y(t+h)-y(t)=\int^{t+h}_tf(y(\tau))d\tau$

\textbf{Trapezoidal Method}: $y_{n+1}=y_n+\frac{h}{2}(f(y_n)+f(y_{n+1}))$ \quad \quad \textbf{Midpoint Method}: {\small $y_{n+1}=y_n+hf(\frac{y_n+y_{n+1}}{2})$}

\textbf{One-Step Collocation Methods (By Lagrange Interpolating Polynomials)}:

\begin{enumerate}[itemsep=-2pt, topsep=-2pt]
    \item \textbf{Lagrange Interpolating Polynomials}: $\ell_{i}(x)=\prod_{j=1,j\ne i}^{s}\frac{x-c_j}{c_i-c_j}\in\mathbb{P}_{s-1}$ \quad where $c_i\in F\in\{\mathbb{Q},\mathbb{R},\mathbb{C}\}$ \\
    $\Rightarrow$ \textbf{Polynomial Interpolation}: $\forall p(x)\in \mathbb{P}_s$ with $p(c_i)=g_i\in F$ \ \ $\Rightarrow$ \ \ $\exists!$ $p(x)=\sum_{i=1}^{s}g_i\ell_i(x)$ {\tiny (Can be proved by Honour Algebra)}
    \item \textbf{Quadrature Rule}: {\small If $g(t)\in\mathbb{P}_{p-1}$} \ \ $\big|$ \ \ {\small $\int^{t_0+h}_{t_0}g(t)dt=\int^1_0g(t_0+hx)dx\approx h\sum_{i=1}^{s}b_ig(t_0+hc_i),b_i:=\int^1_0\ell_i(x)dx$} \quad {\tiny ps: $c_i$ 从 $[0,1]$中取不同的}
    \item \textbf{Collocation Methods}: For: $y(t_0)=y_0 \ , \ y'(t_0+c_ih)=f(y(t_0+c_ih))$ {\tiny ps: $c_i$ 从 $[0,1]$中取不同的} \quad {\small Let: $a_{ij}:=\int_0^{c_i}\ell_j(x)dx$ and $b_i:=\int_0^1\ell_i(x)dx$}\\
    $\Rightarrow$ \ $F_i=f(y_n+h\sum_{j=1}^{s}a_{ij}F_j)$ and $y_{n+1}=y_n+h\sum_{i=1}^{s}b_iF_i$ \hspace{30pt} where $F_i:=y'(t_0+c_ih)$
\end{enumerate}

$\cdot$ \textbf{Remark}: {\footnotesize For choice of $c_i$: The optimal choice is attained by \textit{Gauss-Legendre collocation methods}}.

\textbf{Runge-Kutta Methods}: Let $y'=f(t,y)$ \quad \textbf{Stage Values}: $Y_i=y_n+h\sum_{j=1}^{s}a_{ij}f(Y_{j})$ \ \ $i\in\{1,...,s\}$ \quad $F_i=f(Y_i)$

\begin{enumerate}[itemsep=-2pt, topsep=-2pt]
    \item The RK method is the form: $y_{n+1}=y_n+h\sum_{i=1}^{s}b_if(Y_i(y_n,h))$ \quad for some values of $b_i,a_{ij},s,c_i$ \quad { for Autonomous: $c_i=\sum_{j=1}^{s}a_{ij}$}
    \item Flow-map: $\Psi_{h}(y)=y+h\sum_{i=1}^{s}b_if(Y_{i}(y,h))$ \quad \quad \quad { ps:\textit{weights}: $b_i$; \ \textit{internal coefficients}: $a_{ij}$}
    \item \small{{ We can using \textbf{Butcher Table} to represent the RK method (Appendix)} \quad \textbf{Explicit}: $a_{ij}=0$ for $j\geq i$ { (严格下三角行)} \quad \textbf{Implicit}: $\exists a_{ij}\ne0$ for $j\geq i$ { (Not Explicit)}}
\end{enumerate}


\subsection{Accuracy of RK Method | Order Condition} % Accuracy of RK Method | Order Condition

\textbf{Some Notations}: If $\mathbf{y}=f'(\mathbf{y})$ where $f(\mathbf{y}):\mathbb{R}^d\to\mathbb{R}^d$. \quad Def $f'=(\frac{\partial f_i}{\partial y_j})$,{\scriptsize $1\leq i\leq d,1\leq j\leq d$} \ {\tiny (行向量)} \qquad $f''=(\frac{\partial^2 f_i}{\partial y_j\partial y_k})$,{\scriptsize $1\leq i\leq d,1\leq j,k\leq d$} 

$\cdot$ Def: {\footnotesize $f''(\mathbf{a},\mathbf{b})=\sum_{j=1}^{d}\sum_{k=1}^{d}\frac{\partial^2 f_i}{\partial y_{j} \ \partial y_{k}}a_jb_k$ \quad $\big|$ $y'=f$ \quad $y''=\sum_{j=1}^{d}\frac{\partial f_i}{\partial y_j}f_j=f'f$ \quad $y'''=\sum_{j=1}^{d}\sum_{k=1}^{d}\frac{\partial^2f_i}{\partial y_j \ \partial y_k}y'_j(t)y'_k(t)+\sum_{j=1}^{d}\frac{\partial f_i}{\partial y_j}y''_j(t)=f''(f,f)+f'f'f$}

$\cdot$ $\Phi_h(y)=y+hf+\frac{h^2}{2}f'f+\frac{h^3}{6}[f''(f,f)+f'f'f]+\mathcal{O}(h^4)$

\textbf{Order Condition}: {\small RK method: $y_{n+1}=y_n+h\sum_{i=1}^{s}b_if(Y_i)$, Let $z(h)=\Phi_h(y)$ 

\hspace{110pt} $\Rightarrow$ If $z'(0)=y',z''(0)=y'',...,z^{(n)}=y^{(n)}$ $\Rightarrow$ { \textbf{Convergent at order $n$}}}

$\cdot$ Order 1: $\sum_{i=1}^{s}b_i=1$ \qquad Order 2: { (add)} $\sum_{i=1}^{s}b_ic_i=\frac{1}{2}$ \qquad Order 3: { (add)} $\sum_{i=1}^{s}b_ic_i^2=\frac{1}{3}$ \ and \ $\sum_{i=1}^{s}\sum_{j=1}^{s}b_ia_{ij}c_j=\frac{1}{6}$


\section{Stability of Runge-Kutta Methods { consider for autonomous $y'=f(y)$}} % Stability of Runge-Kutta Methods

\subsection{Basic Definition for Stability} % Basic Definition for Stability

\textbf{Fixed Point-Exact}: For ODEs $\frac{dy}{dt}=f(y)$, point $y^*$ is fixed point if $f(y^*)=0$ $\Leftrightarrow$ $\Phi_t(y^*)=y^*$ {\footnotesize \ \ \textbf{Set of Fixed Points}: $\mathcal{F}=\{y^*\in\mathbb{R}^d:f(y^*)=0\}$}

\textbf{Fixed Point-Numerical}: \textit{One-step} method $\Psi_h(y)$, point $y^*$ is fixed point if $y^*=\Psi_h(y^*)$ {\footnotesize \ \ \textbf{Set of Fixed Points}: $\mathcal{F}_h=\{y^*\in\mathbb{R}^d:y^*=\Psi_h(y^*)\}$}

\textbf{Theorem}: For Runge-Kutta method, $\mathcal{F}\subseteq\mathcal{F}_h$ \qquad \qquad \textbf{Remark}: $\mathcal{F}_h\subseteq\mathcal{F}$ is NOT always true.

$\cdot$ the point in $\mathcal{F}_h\setminus\mathcal{F}$ is called \textbf{spurious fixed point}. \qquad \qquad As $h\to\infty$, the \textit{spurious} fixed points will tends to infinity.

\textbf{Stability of Fixed Points}: Fixed point $y^*$, the ODEs $\frac{dy}{dt}=f(y)$ with $y(0)=y_0$.
\begin{enumerate}[itemsep=-2pt, topsep=-2pt]
    \item \textbf{Stable in the sense of Lyapunov}: Fixed point $y^*$ is stable if $\forall \varepsilon>0,\exists \delta>0$ s.t. $||y_0-y^*||<\delta\Rightarrow||y(t;y_0)-y^*||<\varepsilon$ $\forall t>0$
    \item \textbf{Asymptotically Stable}: Fixed point $y^*$ is asymptotically stable if $\exists \delta>0$ s.t. $||y_0-y^*||<\delta\Rightarrow\lim\limits_{t\to\infty}||y(t;y_0)-y^*||=0$
    \item \textbf{Unstable}: Fixed point $y^*$ is unstable if it's not stable. \quad i.e. $\exists \epsilon>0,\forall \delta>0$ s.t. $||y_0-y^*||<\delta\Rightarrow||y(t)-y^*||\geq\varepsilon$ for some $t$.
\end{enumerate}


\subsection{Classification of Fixed Points} % Classification of Fixed Points

\textbf{Linearization Theorem}: Suppose $\frac{dy}{dt}=f(y)$, $y^*$ is a fixed point. Let $J=f'(y^*)$ be the Jacobian matrix of $f$ at $y^*$.

\begin{enumerate}[itemsep=-2pt, topsep=-2pt]
    \item If $\forall$ eigenvalues of $J$ in left complex half plane, then $y^*$ is \textbf{asymptotically stable}.
    \item If $\exists$ eigenvalues of $J$ in right complex half plane, then $y^*$ is \textbf{unstable}.
\end{enumerate}

(Following is a special cases from HDE)

\textbf{Generalized Eigenvectors}: If $\lambda$ is an repeated eigenvalue with eigenvalue $\xi$ then:

\quad Generalized Eigenvectors: $\eta$ s.t. $(A-\lambda I)\eta=\xi$ \quad \quad More generallyL $(A-\lambda I)\eta_n=\eta_{n-1}$

\textbf{Classification of Critical Points at $y^*$ (Linear)}: \ \ \ \ $r_1,r_2$ be sol of $det(J-\lambda I)=0$. \ \ || \ \ $\mathbb{C}:r=\lambda\pm i\mu(\mu>0)$ 

If $J$ constant, write sol: $\mathbf{x}=c_1e^{r_1t}\xi_1+c_2e^{r_2t}\xi_2$ \ \ || \ \ $GM=1$: $\mathbf{x}=c_1e^{rt}\xi+c_2e^{rt}(t\xi+\eta)$ \qquad {\tiny $J=\begin{pmatrix}\partial_xF(\mathbf{x}_0) & \partial_yF(\mathbf{x}_0) \\ \partial_xG(\mathbf{x}_0) & \partial_yG(\mathbf{x}_0) \\ \end{pmatrix}$} {\tiny If $f(x,y)=\begin{pmatrix}F(x,y) \\ G(x,y)\end{pmatrix}$}

{\tiny
\vspace{-8pt}
\begin{longtable}{|c|l|l|l|l|l|}
    \hline
    {\tiny $\mathbb{R} / \mathbb{C}$} & { Condition || Stability} & Type || Name & Phase Plane Description & Other &  \\
    \hline
    \multirow{7}{*}{$\mathbb{R}$} & { $r_1<r_2<0$ || asy.stab} &  N || NSk & {\tiny 向原点,$\xi_2$直线,$\xi_1$曲线,and $\xi_1$周围$y=\pm x^3$} & { $c_2\ne0,t\to\infty$:$\xi_2$主导方向; \ $c_2=0,t\to\infty$:$\xi_1$主导方向} & \multirow{10}{*}{\tiny \makecell{PS: \\ \textbf{N} = Node \\ \textbf{PN} = Proper Node \\ \textbf{IN} = Improper \\ or: Degenerate Node \\ \textbf{SP} = Saddle Point \\ \textbf{SpP} = spiral point \\ or: Focus Point \\ \textbf{C} = Center \\ \textbf{NSk} = Nodal Sink \\ \textbf{NSo} = Nodal Source}} \\
    \cline{2-5}
    & { $r_1>r_2>0$ || unstable} &  N || NSo & {\tiny 原点向外,$\xi_2$直线,$\xi_1$曲线,and $\xi_1$周围$y=\pm x^3$} & { $c_1\ne0,t\to\infty$:$\xi_1$主导方向; \ $c_1=0,t\to\infty$:$\xi_2$主导方向} & \\
    \cline{2-5}
    & { $r_1>0>r_2$ || unstable} & SP || SP & \makecell{ $t\to\infty$,$\xi_1$从原点向外,$\xi_2$从外向原点 \\  and: 像$y=\pm\frac{1}{x}$,同进同出} & \makecell{ $t\to\pm\infty:|\mathbf{x}|\to\infty$; \ \ \ \ \ $t\to\infty:c_1,c_2\ne0,|\mathbf{x}|\to\infty$,$\xi_1$主导; \ \ \ \ \ \ \ \ \ \ \ \ \ \ \ \ \ \\  $t\to\infty:c_2=0,|\mathbf{x}|\to\infty$,$\xi_1$主导; \ \ \ \ $t\to\infty:c_1=0,|\mathbf{x}|\to0$,$\xi_2$主导} & \\
    \cline{2-5}
    & { $r_1=r_2<0$, {\tiny GM=2} || {\tiny asy.stab}} &  PN || { PN or Stable Star} & { \underline{直线}向原点} & { 直线, $u_1/u_2$ is $t$ independent} & \\
    \cline{2-5}
    & { $r_1=r_2>0$, {\tiny GM=2} || {\tiny unstable}} &  PN || { PN or Unstable Star} & { \underline{直线}从原点向外} & { 直线, $u_1/u_2$ is $t$ independent} & \\
    \cline{2-5}
    & { $r_1=r_2<0$, {\tiny GM=1} || {\tiny asy.stab}} &  IN {\tiny(AL:Type: SpP)} || { IN {\tiny (Stable)}} & { S曲线,向原点} & { $t\to\infty,|\mathbf{x}|\to0,\xi$主导 \ \ \ \ \ \ ps:旋转方向大体和$\eta+c_2\xi$方向相同} & \\
    \cline{2-5}
    & { $r_1=r_2>0$, {\tiny GM=1} || {\tiny unstable}} &  IN {\tiny(AL:Type: SpP)} || { IN {\tiny (Unstable)}} & { S曲线,从原点向外} & { $t\to\infty,|\mathbf{x}|\to\infty,\xi$主导 \ \ \ \ \ \ ps:旋转方向大体和$\eta+c_2\xi$方向相同} & \\
    \cline{1-5}
    \multirow{3}{*}{$\mathbb{C}$} & { $\lambda\ne0,\lambda>0$ || unstable} &  SpP || Unstable Focus & { 向外椭圆(elliptical)螺旋} & { $t\to\infty,|\mathbf{x}|\to\infty$ \ \ \ {\tiny ps:考虑$J=(a,b;c,d)$,如果bc>0,顺时针,如果bc<0,逆时针}} & \\
    \cline{2-5}
    & { $\lambda\ne0,\lambda<0$ || asy.stab} &  SpP || Stable Focus & { 向内椭圆(elliptical)螺旋} & { $t\to\infty,|\mathbf{x}|\to0$ \ \ \ {\tiny ps:考虑$J=(a,b;c,d)$,如果bc>0,顺时针,如果bc<0,逆时针}} & \\
    \cline{2-5}
    & { $\lambda=0$ || stable {\tiny (AL:Indeterminate)}} &  C { (AL:C or SpP)} || C & { 椭圆(elliptical) and 半长轴$\xi$实部方向} & { Bounded trajectory \ \ \ or \ \ \ $\exists$ Periodic Trajectories} & \\
    \hline
\end{longtable}
\vspace{-0.4cm}
}


\subsection{Stability of Fixed Points of Maps (Numerical)}

\textbf{Definition}: For flow map $\Psi$ from $\mathbb{R}^d\to\mathbb{R}^d$. Def $y^{n}(y_0):=$ the $n$-th iterate of $y_0$ under $\Psi$. \ \ {\footnotesize i.e. $y^{n}=y_n$ ; $y_n=\Psi(y_{n-1})$}

\textbf{Stability of Fixed Points of Maps}: Fixed point $y^*$, the map $\Psi$ with $y^*=\Psi(y^*)$.
\begin{enumerate}[itemsep=-2pt, topsep=-2pt]
    \item \textbf{Stable in the sense of Lyapunov}: $y^*$ is stable if $\forall \varepsilon>0,\exists \delta>0$ s.t. $||y_0-y^*||<\delta\Rightarrow||y^{n}(y_0)-y^*||<\varepsilon$ $\forall n\geq0$
    \item \textbf{Asymptotically Stable}: $y^*$ is asymptotically stable if $\exists \delta>0$ s.t. $||y_0-y^*||<\delta\Rightarrow\lim\limits_{n\to\infty}||y^{n}(y_0)-y^*||=0$
    \item \textbf{Unstable}: $y^*$ is unstable if it's not stable. \quad i.e. $\exists \epsilon>0,\forall \delta>0$ s.t. $||y_0-y^*||<\delta\Rightarrow||y^{n}(y_0)-y^*||\geq\varepsilon$ for some $n$.
\end{enumerate}

\textbf{Spectral Radius}: For matrix $K$, $\rho(K)=\max\{|\lambda|:\lambda \ \text{is eigenvalue of} \ K\}$

\textbf{Theorem|Spectral Radius}: Let $z_n=||K^ny_0||$, where $K\in\mathbb{R}^{d\times d}$ is the matrix. Then:

\begin{enumerate}[itemsep=-2pt, topsep=-2pt]
    \item $\rho(K)<1$ $\Leftrightarrow$ $\lim\limits_{n\to\infty}z_n=0$
    \item $\rho(K)>1$ $\Leftrightarrow$ $\lim\limits_{n\to\infty}z_n=\infty$
    \item If $\rho(K)=1$ and \textit{eigenvalues} of $K$ are \textit{semisimple} {\scriptsize (i.e. No generalized eigenvector)}, then $\{z_n\}$ is bounded.
\end{enumerate}

\textbf{Theorem|Connect to Stability}: For map $\Psi$, $y^*=\Psi(y^*)$. Let $K=\Psi'(y^*)$, then: 

\begin{enumerate}[itemsep=-2pt, topsep=-2pt]
    \item $\rho(K)<1$ $\Rightarrow$ $y^*$ is \textit{asymptotically stable}
    \item $\rho(K)>1$ $\Rightarrow$ $y^*$ is \textit{unstable}
\end{enumerate}


\subsection{Linear Stability of Numerical Methods} % Linear Stability of Numerical Methods

\textbf{Special Case|Euler Method}: For $\frac{dy}{dt}=By$, the Euler method is $y_{n+1}=(I+hB)y_n$. where $\lambda_i$ is eigenvalues of $B$.
\begin{enumerate}[itemsep=-2pt, topsep=-2pt]
    \item The origin is \textit{stable} if $||I+h\lambda_i||\leq1$ $\forall i$ \qquad \textit{asymptotically stable} if $||I+h\lambda_i||<1$ $\forall i$
    \item The origin is \textit{unstable} if $||I+hB||>1$ \qquad {\footnotesize ps: 即$h\lambda_i$在 复平面上以$z=-1$为圆心,半径为1的圆内$\leftarrow$称为\textbf{Region of absolute stability}}
\end{enumerate}

\textbf{Stability function $R,P$}: Let $P$ be polynomial function and $R$ be rational function.

\quad {\small If RK is \textit{explicit}, then $y_{n+1}=P(\mu)y_n$ \quad ; \quad If RK is \textit{implicit}, then $y_{n+1}=R(\mu)y_n$ \qquad where $\mu=h\lambda$}

\textbf{Stability function $R(\mu)$|Special Case}: For $\frac{dy}{dt}=\lambda y$ \quad {\footnotesize All RK methods can be written as:} \quad {\footnotesize where: $b^T,A$ are from \textit{Butcher Table}. \ \ $\mathbf{1}=[1,...,1]^T$}

\quad \textbf{I}.$Y_i=y_n+\mu\sum^s_{j=1}a_{ij}Y_j$ \quad ($Y=y_n\mathbf{1}+\mu AY$) \qquad $y_{n+1}=y_n+\mu\sum^s_{b=1}b_iY_j=y_n+\mu b^TY$

\quad \textbf{II}.$R(\mu)=1+\mu b^T(I-\mu A)^{-1}\mathbf{1}$ \hspace{80pt} \textbf{III}. $y_{n+1}=R(\mu)y_n$ \qquad where $\mu=h\lambda$

\textbf{Stability function $R(\mu)$|General}: For $\frac{dy}{dt}=By$ \quad {\footnotesize where: $b^T,A$ are from \textit{Butcher Table}. \ \ $\Lambda,U$是B的特征值分解$U^{-1}BU=\Lambda$} \quad {\scriptsize 此时$z_n,y_n$ 是向量}

\quad \textbf{I}. Let $y_n=U z_n$ and $Y_i=UZ_i$:

\qquad Then $Z_{i}=z_n+h\sum_{j=1}^{s}a_{ij}\Lambda Z_j$ \quad {\footnotesize ($Z_j^{(i)}=z_n^{(i)}\mathbf{1}+\mu AZ^{(i)}_j$ \ \ $\forall i$)} \qquad $z_{n+1}=z_n+h\sum_{i=1}^{s}b_i\Lambda Z_i$ \ \ {\footnotesize ($z_{n+1}^{(i)}=z_{n}^{(i)}+\mu\sum^s_{j=1}b_jZ_j^{(i)}$)}

\quad \textbf{II}. $\frac{dz}{dt}=\Lambda z$ \quad $\Rightarrow$ \quad $\frac{dz^{(i)}}{dt}=\lambda_i z^{(i)}$ \quad $\Rightarrow$ \quad $z_{n+1}^{(i)} = R(\mu)z_n^{(i)}$ \quad where $\mu=h\lambda_i$ \qquad {\footnotesize (回到前一个)}

\textbf{Theorem}: For $\frac{dy}{dt}=By$ with $\lambda_1,...,\lambda_d$ be eigenvalues of $B$. The RK method is \textit{stable}|\textit{asy.stab} at \textit{origin} iff:

\quad The Same method also \textit{stable}|\textit{asy.stab} at \textit{origin} for $\frac{dz}{dt}=\lambda_iz$ $\forall i$

\textbf{Corollary}: For $\frac{dy}{dt}=By$ with $B$ diagonalizable. \ \ An RK Method with \textit{stability function} $R(\mu)$ is \textit{stable}|\textit{asy.stab}|\textit{unstable} at \textit{origin} iff:

\quad $|R(\mu)|\leq1$ \ \ or \ \ $|R(\mu)|<1$ \ \ or \ \ $|R(\mu)|>1$ \quad $\forall \mu=h\lambda_i$ $\forall i$ \qquad {\scriptsize we can write $\sigma(B)=\{\lambda_1,...,\lambda_d\}$ the set of eigenvalues of $B$}

\quad \textbf{Remark}: 这里的$R(\mu)$是指$B$分解后的每一个特征值$\lambda_i$的$R(\mu)$,而不是$B$的$R(\mu)$


\subsection{Stability Region and A-stability} % Stability Region and A-stability

\textbf{Stability Region}: {\small For $\frac{dy}{dt}=By$. An RK method, the \textit{stability region} is the set of $\mu$ where $\widehat{R}(\mu)=|R(\mu)|<1$.} {\tiny (如$y$是向量,$R(\mu)$按上面corollary的remark所说)}

\begin{enumerate}[itemsep=-2pt, topsep=-2pt]
    \item Euler's Method: $\widehat{R}(\mu)=|1+\mu|$ \quad $\Rightarrow$ \quad $\mu\in\{z\in\mathbb{C}:|1+z|<1\}$ {\scriptsize (-1处半径为1的圆)}
    \item Trapezoidal Rule: $\widehat{R}(\mu)=\left|\frac{1+\mu/2}{1-\mu/2}\right|$ \quad $\Rightarrow$ \quad $\mu\in\{z\in\mathbb{C}:|1+z/2|<|1-z/2|\}$ {\scriptsize (left complex half-plane, A-stable)}
    \item Implicit Euler: $\widehat{R}(\mu)=|1-\mu|^{-1}$ \quad $\Rightarrow$ \quad $\mu\in\{z\in\mathbb{C}:|1-z|>1\}$ {\scriptsize (-1处半径为1的圆外侧)}
    \item RK4: $\widehat{R}(\mu)=\left|1+\mu+\frac{\mu^2}{2}+\frac{\mu^3}{6}+\frac{\mu^4}{24}\right|$ \quad $\Rightarrow$ \quad Using $R(\mu)=e^{i\theta}$ to find the region.
\end{enumerate}

\textbf{A-Stable}: An RK method is \textit{A-stable} if its \textit{stability region} contains the entire \textit{left complex half-plane}. {\scriptsize (i.e. $\Re(z)<0$)}


\section{Appendix} 

\subsection{Useful Series | Common RK Methods} % Useful Series | Common RK Methods
\small{
$e^x = 1 + x + \frac{x^2}{2!} + \frac{x^3}{3!} + \cdots  = \sum_{n=0}^\infty \frac{x^n}{n!}$ \quad $ \sin(x) = x - \frac{x^3}{3!} + \frac{x^5}{5!} + \cdots = \sum_{n=0}^\infty (-1)^n \frac{x^{2n+1}}{(2n+1)!} $ \quad $\cos(x) = 1 - \frac{x^2}{2!} + \frac{x^4}{4!} + \cdots = \sum_{n=0}^\infty (-1)^n \frac{x^{2n}}{(2n)!}$

$\ln(1+x) = x - \frac{x^2}{2} + \frac{x^3}{3}  + \cdots = \sum_{n=1}^\infty (-1)^{n+1} \frac{x^n}{n}$ \quad $\arctan(x) = x - \frac{x^3}{3} + \frac{x^5}{5} + \cdots = \sum_{n=0}^\infty (-1)^n \frac{x^{2n+1}}{2n+1}$ \quad $\sinh(x) = x + \frac{x^3}{3!} + \frac{x^5}{5!} + \cdots = \sum_{n=0}^\infty \frac{x^{2n+1}}{(2n+1)!}$

$\cosh(x) = 1 + \frac{x^2}{2!} + \frac{x^4}{4!} + \cdots = \sum_{n=0}^\infty \frac{x^{2n}}{(2n)!}$  \quad $(1+x)^k = 1 + kx + \frac{k(k-1)x^2}{2!} + \cdots = \sum_{n=0}^\infty \binom{k}{n} x^n$ $\frac{1}{1-x} = 1 + x + x^2  + \cdots = \sum_{n=0}^\infty x^n $ 

$\frac{1}{1+x} = 1 - x + x^2  + \cdots = \sum_{n=0}^\infty (-1)^n x^n$ \quad $\ln(x) = (x-1) - \frac{(x-1)^2}{2} - \cdots = \sum_{n=1}^\infty (-1)^{n+1} \frac{(x-1)^n}{n}, \, x > 0$
}

\textbf{Common Runge-Kutta Methods (Butcher Table)}:

\vspace{-5pt}
\begin{minipage}{0.15\linewidth}
    \centering
    \begin{tabular}{c|ccc}
        $c_1$ & $a_{11}$ & $\cdots$ & $a_{1s}$ \\
        $\vdots$ & $\vdots$ & $\ddots$ & $\vdots$ \\
        $c_s$ & $a_{s1}$ & $\cdots$ & $a_{ss}$ \\
        \hline
            & $b_1$ & $\cdots$ & $b_s$
    \end{tabular}
    \\~\\
    {Example}
\end{minipage}
\hfill
\begin{minipage}{0.15\linewidth}
    \centering
    \begin{tabular}{c|c}
        0 &  \\
        \hline
          & 1
    \end{tabular}
    \\~\\
    {RK1 \\ (Euler's Method)}
\end{minipage}
\hfill
\begin{minipage}{0.15\linewidth}
    \centering
    \begin{tabular}{c|cc}
        0   &    &  \\
        1   & 1   &  \\
        \hline
            & 1/2 & 1/2
    \end{tabular}
    \\~\\
    {RK2 (Heun's Method)}
\end{minipage}
\hfill
\begin{minipage}{0.2\linewidth}
    \centering
    \begin{tabular}{c|ccc}
        0   &    &    &  \\
        1/2 & 1/2 &    &  \\
        1   & -1  & 2   &  \\
        \hline
            & 1/6 & 2/3 & 1/6
    \end{tabular}
    \\~\\
    {RK3}
\end{minipage}
\hfill
\begin{minipage}{0.3\linewidth}
    \centering
    \begin{tabular}{c|cccc}
        0   &    &    &    &  \\
        1/2 & 1/2 &    &    &  \\
        1/2 & 0   & 1/2 &    &  \\
        1   & 0   & 0   & 1   &  \\
        \hline
            & 1/6 & 1/3 & 1/3 & 1/6
    \end{tabular}
    \\~\\
    {RK4 (Classical/Famous)}
\end{minipage}


\end{document}
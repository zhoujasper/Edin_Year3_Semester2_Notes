\documentclass[9pt]{article}

\usepackage{graphicx} % Required for inserting images
\usepackage{amsmath}
\usepackage{amsfonts}
\usepackage{ctex}
\usepackage{enumitem}
\usepackage{longtable}
\usepackage{makecell} % 换行

% 使用分栏宏包
\usepackage{multicol} 
\setlength{\columnseprule}{0.4pt} % 分割线

% 设置字体
\usepackage{unicode-math}
\setmainfont{Cambria}
\setmathfont{Cambria Math}

% 调整页面布局
\usepackage[a4paper, top=0.7cm, bottom=1cm, left=0.7cm, right=.7cm]{geometry}
\setlength{\footskip}{15pt}

% 设置页脚/页眉
\usepackage{fancyhdr}
\fancyfoot[C]{Copyright By Jingren Zhou | Page \thepage}
\fancyhead[]{}
\pagestyle{fancy}
% 去除线
\renewcommand{\headrulewidth}{0pt}
\renewcommand{\footrulewidth}{0pt}

% 设置 section/subsection 之间的行间距
\usepackage{titlesec}
\titlespacing*{\section}{0pt}{0pt}{0pt}
\titlespacing*{\subsection}{0pt}{0pt}{0pt}

% 调整标题上下间距
\usepackage{titling}
\setlength{\droptitle}{-2.4cm} % 负值表示向上移动

% 设置标题,作者,时间
\title{NODEA Note}
\author{}
\date{}

% 正文
\begin{document}


% 标题
\maketitle
\thispagestyle{fancy}
\vspace{-3.5cm}

% 字体大小
\fontsize{10pt}{11pt}\selectfont
\setlength{\parindent}{8pt}


\section{Basic Knowledge} % Basic Knowledge

\textbf{Def of ODE \& ODEs}: {\scriptsize (1st order)} ODE: $\frac{dy}{dt}=f(t,y)$ \quad \& \quad ODEs: $\frac{d\mathbf{y}}{dt}=\mathbf{f}(t,\mathbf{y})$ , {\footnotesize $\mathbf{y}=(y_1,...,y_d)^T,\mathbf{f}(t,\mathbf{y})=(f_1(t,\mathbf{y}),...,f_d(t,\mathbf{y}))^T$}

\textbf{Autonomous}: $\frac{d\mathbf{y}}{dt}=\mathbf{f}(\mathbf{y})$ \ $\Rightarrow$ \ autonomous ODE(s). \quad \quad \quad || $\Downarrow$ New Autonomous ODEs: $\frac{d\mathbf{y}}{ds}=\mathbf{f}(y_{d+1},\mathbf{y})$ and $\frac{dy_{d+1}}{ds}=1$

$\cdot$ \textbf{Change to Autonomous}: For $\frac{d\mathbf{y}}{dt}=\mathbf{f}(t,\mathbf{y})$. Let $y_{d+1}=t$ and \textit{new} independent variable $s$ s.t. $\frac{dt}{ds}=1$ $\Uparrow$

\textbf{Linearity}: ODE: $\frac{dy}{dt}=f(t,y)$ is linearity if $f(t,y)=a(t)y+b(t)$ \quad || ODEs: {\small If each ODE is linear, then the ODEs are linear.}

\textbf{Picard's Theorem}: If $f(t,y)$ is continuous in $D:=\{(t,y):t_0\leq t\leq T,|y-y_0|<K\}$ and $\exists L>0$ {\footnotesize (Lipschitz constant)} s.t.

$\quad$ $\forall (t,u),(t,v)\in D \quad|f(t,u)-f(t,v)|\leq L|u-v|$ {\tiny (ps:Can use MVT)}. And Assume that $M_f(T-t_0)\leq K,M_f:=\max\{|f(t,u)|:(t,u)\in D\}$

$\quad$ $\Rightarrow$ \textbf{Then}, $\exists$ a unique continuously differentiable solution $y(t)$ to the IVP $\frac{dy}{dt}=f(t,y),y(t_0)=y_0$ on $t\in[t_0,T]$.

\textbf{Existence \& Uniqueness Theorem}: IVP $\frac{d\mathbf{y}}{dt}=\mathbf{f}(t,\mathbf{y}),\mathbf{y}(t_0)=\mathbf{y}_0$. If $f(t,y)$ and $\frac{\partial f}{\partial y_i}$ are continuous in a neighborhood of $(t_0,\mathbf{y}_0)$.

$\quad$ $\Rightarrow$ \textbf{Then}, $\exists I:=(t_0-\delta,t_0+\delta)$ s.t. $\exists$ a unique continuously differentiable solution $\mathbf{y}(t)$ to the IVP on $t\in I$.


\section{Acknowledge} % Acknowledge

\vspace{-10pt}
\begin{longtable}{|c|l||c|l|}
    \hline
    Notation & Meaning & Notation & Meaning \\
    \hline
    \hline
    $[a,b]$ & Approximate function for $t\in[a,b]$ & {\footnotesize $t_0=a \ | \ t_N=b$} & Assume that $t_0=a,t_N=b$ \\
    \hline
    $N$ & number of \textbf{timesteps} \ \ {\tiny (i.e. Break up interval $[a,b]$ into $N$ equal-length sub-intervals)} & $h$ & \textbf{stepsize} \ \ {\scriptsize ($h=\frac{b-a}{N}$)} \\
    \hline
    $t_i$ & Define $N+1$ points: $t_0,t_1,...,t_N$ & $t_m$ & $t_m=a+h\cdot m=t_0+h\cdot m$ \\
    \hline
    $y_i$ & Approximation of $y$ at point $t=t_i$ \ \ (Except $y_0$) & $y(t_i)$ & Exact value of $y$ at point $t=t_i$ \\
    \hline
\end{longtable}
\vspace{-10pt}


\section{Euler's Method and Taylor Series Method} % Euler's Method and Taylor Series Method

\textbf{Euler's Method Algorithm}: Approximate ODE $\frac{dy}{dt}=f(t,y),y(t_0)=y_0$ with number of steps $N$. \ (Similarly for ODEs)

\hspace{20pt} $\Rightarrow$
\textbf{for} $n=0,1,2,...,N-1$:
\quad $y_{n+1}=y_n+hf(t_0+nh,y_n)$ \quad 
\textbf{end} \hspace{100pt} {\scriptsize (ps: $\Downarrow$ Can get $|y''|<M$)}

\textbf{Boundedness Theorem}: {\small For $\frac{dy}{dt}=f(t,y),y(a)=y_0$ and suppose there exists a unique, twice differentiable, solution $y(t)$ on $[a,b]$.}

\quad \quad \quad {\small Suppose: $y$ is continuous and $|\frac{\partial f}{\partial y}|\leq L$. \ $\Rightarrow$ \ the solution $y_n$ given Euler's method satisfies: $e_n=|y_n-y(t_n)|\leq Dh,D=e^{(b-a)L}\frac{M}{2L}$}

$\cdot$ \textbf{Lemma}: If $v_{n+1}\leq Av_n+B$, then $v_n\leq A^nv_0+\frac{A^n-1}{A-1}B$ \quad If $v_{n}=e_n:=y_n-y(t_n)$, then $A=1+hL,B=h^2M/2$ {\scriptsize (suppose $|y''|<M$)}

\textbf{Order Notation ($\mathcal{O}$)}: we write $z=\mathcal{O}(h^p)$ if $\exists C,h_0>0$ s.t. $|z|\leq Ch^p,0<h<h_0$ \quad (i.e. $z$的速率不超过$h^p$)

\textbf{Flow Map ($\Phi$)}: $\Phi_{t_0,h}(y_0):=y(t_0+h;t_0,y_0)$ \quad {\small Determining the flow map is equivalent to solving the IVP $\frac{dy}{dt}=f(t,y),y(t_0)=y_0$}

\quad $\Rightarrow$ Suppose $\frac{dy}{dt}=f(t,y)$. We approximate $\Phi_{t,h}(y)$: $\widehat{\Phi}_{t_0,h}(y_0)= y_0+hf(t_0,y_0)$ \quad {\scriptsize (通过对$y$在$t_0+h$处泰勒展开得到)}




\end{document}
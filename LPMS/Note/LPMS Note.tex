\documentclass[9pt]{article}

\usepackage{graphicx} % Required for inserting images
\usepackage{amsmath}
\usepackage{amsfonts}
\usepackage{ctex}
\usepackage{enumitem}
\usepackage{longtable}
\usepackage{makecell} % 换行

% 使用分栏宏包
\usepackage{multicol} 
\setlength{\columnseprule}{0.4pt} % 分割线

% 设置字体
\usepackage{unicode-math}
\setmainfont{Cambria}
\setmathfont{Cambria Math}

% 调整页面布局
\usepackage[a4paper, top=0.7cm, bottom=1cm, left=0.7cm, right=.7cm]{geometry}
\setlength{\footskip}{15pt}

% 设置页脚/页眉
\usepackage{fancyhdr}
\fancyfoot[C]{Copyright By Jingren Zhou | Page \thepage}
\fancyhead[]{}
\pagestyle{fancy}
% 去除线
\renewcommand{\headrulewidth}{0pt}
\renewcommand{\footrulewidth}{0pt}

% 设置 section/subsection 之间的行间距
\usepackage{titlesec}
\titlespacing*{\section}{0pt}{0pt}{0pt}
\titlespacing*{\subsection}{0pt}{0pt}{0pt}

% 调整标题上下间距
\usepackage{titling}
\setlength{\droptitle}{-2.4cm} % 负值表示向上移动

% 设置标题,作者,时间
\title{LPMS Note}
\author{}
\date{}

% 正文
\begin{document}


% 标题
\maketitle
\thispagestyle{fancy}
\vspace{-3.5cm}

% 字体大小
\fontsize{10pt}{11pt}\selectfont
\setlength{\parindent}{8pt}
\setlength{\abovedisplayskip}{0pt}
\setlength{\belowdisplayskip}{0pt}


\section{General Linear Programming Problem} % General Linear Programming Problem

\textbf{General LP Problem}: {\small \textbf{Decision Variables}: $x_i$ \quad \textbf{parameters}: $a_{ij},b_i,c_i$ \quad \textbf{Objective Function}: $f$ \quad \textbf{Constraints}: {\small subject to 后边的部分}}

\vspace{-9pt}
\begin{multicols}{2}

    LP problem can be written as:
    \begin{align*}
        \text{maximize} \quad \ \ f = & c_1x_1 + c_2x_2 + \cdots + c_nx_n & \\
        \text{subject to} \quad \qquad & a_{11}x_1 + a_{12}x_2 + \cdots + a_{1n}x_n \leq b_1 & \\
                                & a_{21}x_1 + a_{22}x_2 + \cdots + a_{2n}x_n \leq b_2 & \\
                                & \vdots & \\
                                & a_{m1}x_1 + a_{m2}x_2 + \cdots + a_{mn}x_n \leq b_m & \\
                                & x_1 \geq 0, x_2 \geq 0, \cdots, x_n \geq 0 &
    \end{align*}
    
    \columnbreak

    We can also write the LP problem in matrix form:

    \vspace{5pt}
    $
    A=
    \begin{pmatrix}
        a_{11} & \cdots & a_{1n} \\
        \vdots & \ddots & \vdots \\
        a_{m1} & \cdots & a_{mn} \\
    \end{pmatrix}
    ,\mathbf{b}=
    \begin{pmatrix}
        b_1 \\
        \vdots \\
        b_m
    \end{pmatrix}
    ,\mathbf{c}=
    \begin{pmatrix}
        c_1 \\
        \vdots \\
        c_n
    \end{pmatrix}
    ,\mathbf{x}=
    \begin{pmatrix}
        x_1 \\
        \vdots \\
        x_n
    \end{pmatrix}
    $

    \begin{align*}
        \text{maximize} \quad \ \ f = & \mathbf{c}^T\mathbf{x} & \\
        \text{subject to} \quad \qquad & A\mathbf{x} \leq \mathbf{b} & \mathbf{x} \geq 0 \\
    \end{align*}

\end{multicols}

\vspace{-15pt}
\textbf{Feasible Solution}: If $\mathbf{x}$ satisfies all constraints (i.e. $A\mathbf{x}\leq\mathbf{b}$), then $\mathbf{x}$ is a feasible solution. {\footnotesize (可行解)}\quad \textbf{Optimal Sol}: {\footnotesize (最优解) (可多个)}

\textbf{Find Optimal Solution}: \textbf{Graphical Method}: {\footnotesize 略.} \quad \textbf{Vertex Enumeration}: {\footnotesize 所有顶点检查, 找出最优解.} \quad \textbf{Simplex Method}: {\footnotesize 后面详述.}

\textbf{Slack Variables}: For each inequality constraint, we introduce a slack variable $x_i$ ($i>n$) to convert it to an equation. {\footnotesize (松弛变量)}

\vspace{-9pt}
\begin{multicols}{2}

    LP problem can be written as: \quad \quad ps: $x_{i}\geq0$ ($i>n$).
    \begin{align*}
        \text{maximize} \quad \ \ f = & c_1x_1 + c_2x_2 + \cdots + c_nx_n & \\
        \text{subject to} \quad \qquad & a_{11}x_1 + a_{12}x_2 + \cdots + a_{1n}x_n + x_{n+1} = b_1 & \\
                                & a_{21}x_1 + a_{22}x_2 + \cdots + a_{2n}x_n + x_{n+2} = b_2 & \\
                                & \vdots & \\
                                & a_{m1}x_1 + a_{m2}x_2 + \cdots + a_{mn}x_n + x_{n+m} = b_m & \\
                                & x_1 \geq 0, x_2 \geq 0, \cdots, x_{n+m} \geq 0 &
    \end{align*}
    
    \columnbreak

    We can also write the LP problem in matrix form:

    \vspace{5pt}
    $
    \overline{A}= [ \ A  \ \ I_m \ ]
    ,\mathbf{b}=\mathbf{b}
    ,\overline{\mathbf{c}}=
    \begin{pmatrix}
        \mathbf{c} \\
        \mathbf{0}_{\tiny m\times 1}
    \end{pmatrix}
    ,\mathbf{x}=
    \begin{pmatrix}
        \mathbf{x} \\
        \vdots \\
        x_{\tiny n+m}
    \end{pmatrix}
    $

    \begin{align*}
        \text{maximize} \quad \ \ f = & \overline{\mathbf{c}}^T\mathbf{x} & \\
        \text{subject to} \quad \qquad & \overline{A}\mathbf{x} = \mathbf{b} & \mathbf{x} \geq 0 \\
    \end{align*}
    
    \vspace{-20pt}
    \textbf{Solving}: $\mathbf{x}=\mathbf{x}_b+\mathbf{v}$ where $\overline{A}\mathbf{v}=0$ and $\mathbf{x}_b=[\mathbf{0} \ \mathbf{b}]^T$

\end{multicols}

\vspace{-15pt}
\textbf{Feasible Region}: {\small It's the set $K$ of the solutions to $\overline{A}\mathbf{x}=\mathbf{b}$} \quad \textbf{Convex Set}: {\small The set $K$ is convex if $\forall \mathbf{x},\mathbf{x'}\in K$, $\forall\theta\in[0,1],\mathbf{x}_{\theta}=(1-\theta)\mathbf{x}+\theta\mathbf{x'}\in K$.}

\textbf{Vertex on Convex Set}: {\footnotesize A vertex of the convex set $K$ is a point $\mathbf{x}\in K$ which doesn't lie strictly inside any line segment connecting two points in $K$.}

$\cdot$ \textbf{Theorem}: {\small If LP has a unique optimal solution is a vertex.} \quad \quad \textbf{Theorem}: {\small If LP has a non-unique solution, $\exists$ optimal solution at vertex}

\textbf{Basic solution}: {\scriptsize 修改$x_i$顺序: $\overline{A}=[B \ N]$ with corresponding $\mathbf{x}=[\mathbf{x}_B \ \mathbf{x}_N]^T$, where $B$ is $m\times m$ invertible matrix. \quad $\overline{A}\mathbf{x}=\mathbf{b}\Leftrightarrow B\mathbf{x}_B+N\mathbf{x}_N=\mathbf{b}$ \quad $\mathbf{x}_B=B^{-1}\mathbf{b},\mathbf{x}_N=\mathbf{0}$ is basic sol.}

$\cdot$ {\scriptsize We use $\mathcal{B}=$ index of $x_i$ in $B$. \quad $\mathcal{N}=$ index of $x_i$ in $N$} \quad \quad \quad {\footnotesize \textbf{Basic Feasible Solution (BFS)}: If $\mathbf{x}_B\geq0$, it is. \ + \ It's a vertex of $K$.}


\end{document}
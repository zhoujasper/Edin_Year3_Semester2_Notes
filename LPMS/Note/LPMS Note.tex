\documentclass[9pt]{article}

\usepackage{graphicx} % Required for inserting images
\usepackage{amsmath}
\usepackage{amsfonts}
\usepackage{ctex}
\usepackage{enumitem}
\usepackage{longtable}
\usepackage{makecell} % 换行

% 使用分栏宏包
\usepackage{multicol} 
\setlength{\columnseprule}{0.4pt} % 分割线

% 设置字体
\usepackage{unicode-math}
\setmainfont{Cambria}
\setmathfont{Cambria Math}

% 调整页面布局
\usepackage[a4paper, top=0.7cm, bottom=1cm, left=0.7cm, right=.7cm]{geometry}
\setlength{\footskip}{15pt}

% 设置页脚/页眉
\usepackage{fancyhdr}
\fancyfoot[C]{Copyright By Jingren Zhou | Page \thepage}
\fancyhead[]{}
\pagestyle{fancy}
% 去除线
\renewcommand{\headrulewidth}{0pt}
\renewcommand{\footrulewidth}{0pt}

% 设置 section/subsection 之间的行间距
\usepackage{titlesec}
\titlespacing*{\section}{0pt}{0pt}{0pt}
\titlespacing*{\subsection}{0pt}{0pt}{0pt}

% 调整标题上下间距
\usepackage{titling}
\setlength{\droptitle}{-2.4cm} % 负值表示向上移动

% 设置标题,作者,时间
\title{LPMS Note}
\author{}
\date{}

% 正文
\begin{document}


% 标题
\maketitle
\thispagestyle{fancy}
\vspace{-3.5cm}

% 字体大小
\fontsize{10pt}{11pt}\selectfont
\setlength{\parindent}{8pt}
\setlength{\abovedisplayskip}{0pt}
\setlength{\belowdisplayskip}{0pt}


\section{General Linear Programming Problem} % General Linear Programming Problem

\textbf{General LP Problem}: {\small \textbf{Decision Variables}: $x_i$ \quad \textbf{parameters}: $a_{ij},b_i,c_i$ \quad \textbf{Objective Function}: $f$ \quad \textbf{Constraints}: {\small subject to 后边的部分}}

\vspace{-9pt}
\begin{multicols}{2}

    LP problem can be written as:
    \begin{align*}
        \text{maximize} \quad \ \ f = & c_1x_1 + c_2x_2 + \cdots + c_nx_n & \\
        \text{subject to} \quad \qquad & a_{11}x_1 + a_{12}x_2 + \cdots + a_{1n}x_n \leq b_1 & \\
                                & a_{21}x_1 + a_{22}x_2 + \cdots + a_{2n}x_n \leq b_2 & \\
                                & \vdots & \\
                                & a_{m1}x_1 + a_{m2}x_2 + \cdots + a_{mn}x_n \leq b_m & \\
                                & x_1 \geq 0, x_2 \geq 0, \cdots, x_n \geq 0 &
    \end{align*}
    
    \columnbreak

    We can also write the LP problem in matrix form:

    \vspace{5pt}
    $
    A=
    \begin{pmatrix}
        a_{11} & \cdots & a_{1n} \\
        \vdots & \ddots & \vdots \\
        a_{m1} & \cdots & a_{mn} \\
    \end{pmatrix}
    ,\mathbf{b}=
    \begin{pmatrix}
        b_1 \\
        \vdots \\
        b_m
    \end{pmatrix}
    ,\mathbf{c}=
    \begin{pmatrix}
        c_1 \\
        \vdots \\
        c_n
    \end{pmatrix}
    ,\mathbf{x}=
    \begin{pmatrix}
        x_1 \\
        \vdots \\
        x_n
    \end{pmatrix}
    $

    \begin{align*}
        \text{maximize} \quad \ \ f = & \mathbf{c}^T\mathbf{x} & \\
        \text{subject to} \quad \qquad & A\mathbf{x} \leq \mathbf{b} & \mathbf{x} \geq 0 \\
    \end{align*}

\end{multicols}

\vspace{-15pt}
\textbf{Feasible Solution}: If $\mathbf{x}$ satisfies all constraints (i.e. $A\mathbf{x}\leq\mathbf{b}$), then $\mathbf{x}$ is a feasible solution. {\footnotesize (可行解)}\quad \textbf{Optimal Sol}: {\footnotesize (最优解) (可多个)}

\textbf{Find Optimal Solution}: \textbf{Graphical Method}: {\footnotesize 略.} \quad \textbf{Vertex Enumeration}: {\footnotesize 所有顶点检查, 找出最优解.} \quad \textbf{Simplex Method}: {\footnotesize 后面详述.}

\textbf{Slack Variables}: For each inequality constraint, we introduce a slack variable $x_i$ ($i>n$) to convert it to an equation. {\footnotesize (松弛变量)}

\vspace{-9pt}
\begin{multicols}{2}

    LP problem can be written as: \quad \quad ps: $x_{i}\geq0$ ($i>n$).
    \begin{align*}
        \text{maximize} \quad \ \ f = & c_1x_1 + c_2x_2 + \cdots + c_nx_n & \\
        \text{subject to} \quad \qquad & a_{11}x_1 + a_{12}x_2 + \cdots + a_{1n}x_n + x_{n+1} = b_1 & \\
                                & a_{21}x_1 + a_{22}x_2 + \cdots + a_{2n}x_n + x_{n+2} = b_2 & \\
                                & \vdots & \\
                                & a_{m1}x_1 + a_{m2}x_2 + \cdots + a_{mn}x_n + x_{n+m} = b_m & \\
                                & x_1 \geq 0, x_2 \geq 0, \cdots, x_{n+m} \geq 0 &
    \end{align*}
    
    \columnbreak

    We can also write the LP problem in matrix form:

    \vspace{5pt}
    $
    \overline{A}= [ \ A  \ \ I_m \ ]
    ,\mathbf{b}=\mathbf{b}
    ,\overline{\mathbf{c}}=
    \begin{pmatrix}
        \mathbf{c} \\
        \mathbf{0}_{\tiny m\times 1}
    \end{pmatrix}
    ,\mathbf{x}=
    \begin{pmatrix}
        \mathbf{x} \\
        \vdots \\
        x_{\tiny n+m}
    \end{pmatrix}
    $

    \begin{align*}
        \text{maximize} \quad \ \ f = & \overline{\mathbf{c}}^T\mathbf{x} & \\
        \text{subject to} \quad \qquad & \overline{A}\mathbf{x} = \mathbf{b} & \mathbf{x} \geq 0 \\
    \end{align*}
    
    \vspace{-20pt}
    \textbf{Solving}: $\mathbf{x}=\mathbf{x}_b+\mathbf{v}$ where $\overline{A}\mathbf{v}=0$ and $\mathbf{x}_b=[\mathbf{0} \ \mathbf{b}]^T$

\end{multicols}

\vspace{-15pt}
\textbf{Feasible Region}: {\small It's the set $K$ of the solutions to $\overline{A}\mathbf{x}=\mathbf{b}$} \quad \textbf{Convex Set}: {\small The set $K$ is convex if $\forall \mathbf{x},\mathbf{x'}\in K$, $\forall\theta\in[0,1],\mathbf{x}_{\theta}=(1-\theta)\mathbf{x}+\theta\mathbf{x'}\in K$.}

\textbf{Vertex on Convex Set}: {\footnotesize A vertex of the convex set $K$ is a point $\mathbf{x}\in K$ which doesn't lie strictly inside any line segment connecting two points in $K$.}

$\cdot$ \textbf{Theorem}: {\small If LP has a unique optimal solution is a vertex.} \quad \quad \textbf{Theorem}: {\small If LP has a non-unique solution, $\exists$ optimal solution at vertex}


\section{Simplex Method} % Simplex Method

\textbf{Solve LP Problem}: Assume $f=\overline{\mathbf{c}}^T\mathbf{x}$ with $\overline{A}\mathbf{x}=\mathbf{b}$ and $\mathbf{x}\geq0$.

\begin{enumerate}[itemsep=-2pt, topsep=-2pt]
    \item {\footnotesize 修改$x_i$/列顺序,$A$中换顺序后得到可逆的$B_{m\times m}$ \ \ $\Rightarrow$ \ \ $\overline{A}\mathbf{x}=\mathbf{b}\Leftrightarrow B\mathbf{x}_B+N\mathbf{x}_N=\mathbf{b}$ \quad $\overline{\mathbf{c}}\rightarrow[\mathbf{c}_B \ \mathbf{c}_N]^T$ \ \ $\overline{A}\rightarrow[B \ N]$ \ \ $\mathbf{x}\to[\mathbf{x}_B \ \mathbf{x}_N]^T$ \quad Let $\widehat{\mathbf{b}}=B^{-1}\mathbf{b}$ \ \ $\widehat{N}=B^{-1}N$}
    \item \textbf{Solution}: $\mathbf{x}_B=\widehat{\mathbf{b}}-\widehat{N}\mathbf{x}_N$ \quad \quad If $\mathbf{x}_N=0$ $\Rightarrow$ It's a basic solution. {\scriptsize (But we need to check whether $\mathbf{x}_B\geq0$)} \quad \quad {\scriptsize Using $\mathcal{B},\mathcal{N}$: Index set of independent/else.}
    \item \textbf{Basic Variables}: $\mathbf{x}_B$ \quad \textbf{Nonbasic Variables}: $\mathbf{x}_N$
    \item \textbf{At Basic Solution}: $\mathbf{x}_B=\widehat{\mathbf{b}},\mathbf{x}_N=\mathbf{0}$ \quad $\Rightarrow$ \quad If $\widehat{\mathbf{b}}\geq0$. Corresponding $\mathbf{x}$ is: $^1$ vertex of $K$; \ \ $^2$ \textbf{Basic Feasible Solution (BFS)}
    \item \textbf{Basic Costs}: $\mathbf{c}^T_B$ \quad \textbf{Nonbasic Costs}:$\mathbf{c}^T_N$ \quad \textbf{Reduced Costs}: $\widehat{\mathbf{c}}_N=\mathbf{c}_N -\widehat{N}^T\mathbf{c}_B=\mathbf{c}_N-N^TB^{-T}\mathbf{c}_B$ \quad $\widehat{f}=\mathbf{c}_B^T\widehat{\mathbf{b}}$ \quad \quad $\mathbf{x}_B,\mathbf{x}_N\geq0$
    \item \textbf{Objective Value}: $f=\overline{\mathbf{c}}^T\mathbf{x}=\mathbf{c}^T_B\mathbf{x}_B+\mathbf{c}^T_N\mathbf{x}_N=\widehat{f}+\widehat{\mathbf{c}}_N^T\mathbf{x}_N$ \quad \quad If $\widehat{\mathbf{c}}_N\leq0$, then $f\leq\widehat{f}$ \ \ $\Rightarrow$ \ \ Corresponding $\mathbf{x}$ is optimal.
    \item If $\widehat{\mathbf{c}}_N\leq0$ doesn't hold, we using \textbf{Simplex Algorithm}.
\end{enumerate}

\textbf{Simplex Algorithm}:

\begin{enumerate}[itemsep=-2pt, topsep=-2pt]
    \item {\small \textbf{Initial Basic Feasible Solution}: Try $\mathcal{B}=\{n+1,...,n+m\}$ and $\mathcal{N}=\{1,...,n\}$ \ \ $\Rightarrow$ \ \ $B=I,N=A$ , $\mathbf{x}_B=\widehat{\mathbf{b}}=\mathbf{b}$, $\mathbf{x}_N=\mathbf{0}$, $\mathbf{c}_B=\mathbf{0}$, $\mathbf{c}_N=\mathbf{c}$}
    \item If $\mathbf{b}\geq0$. $\Rightarrow$ Basis is feasible + cont. \qquad Else: solved by \textbf{Two-Phase Simplex Method}.
    \item\label{start_simplex} If $\widehat{\mathbf{c}}_N\leq0. \Rightarrow$ Optimal Solution \qquad Else: cont.
    \item Let $q'\in\mathcal{N}$ {\small 对应$\widehat{\mathbf{c}}_N$中最大positive分量的index} \quad {\small 同理,对应的最大正分量值为$\widehat{c_q}$} \quad {\small 对应的variable为$x_{q'}$} \quad {\small 对应$N$中的[$q'$]列为$\mathbf{a}_q$}
    \item Let $\widehat{\mathbf{a}_q}=B^{-1}\mathbf{a_q}$. \qquad If $\widehat{\mathbf{a}_q}\leq0$ $\Rightarrow$ LP is unbounded. \qquad Else: LP is bounded + cont.
    \item Let $p'\in\mathcal{B}$ \ be the index corresponding to $p=\arg\min_{i=1,...,m \ ; \ \widehat{a}_{iq}>0}\frac{\widehat{b_i}}{\widehat{a}_{iq}}$ {\scriptsize $p$是对应的index,not value} \qquad \quad {\small 用$\overline{\alpha}=\frac{\widehat{b_p}}{\widehat{a}_{pq}}$代表值}\\
    $\widehat{b_i},\widehat{a}_{iq}$ {\small 代表$\widehat{\mathbf{b}}$的第$i$个分量, $\widehat{\mathbf{a}_q}$的第$i$个分量} \\
    {\small $p'$代表能使$\frac{\widehat{b_i}}{\widehat{a}_{iq}}$的值最小的index,前提条件是$\widehat{a}_{iq}>0$} \qquad \qquad {对应的variable为$x_{p'}$}
    \item Exchange $p'$ and $q'$ between $\mathcal{B}$ and $\mathcal{N}$ \ \ $\Rightarrow$ \ \ {\small values of new $x_{B}=\widehat{\mathbf{b}}-\overline{\alpha} \ \widehat{\mathbf{a}}_q$} \qquad Update $\mathcal{B},\mathcal{N},B,N,\widehat{\mathbf{b}},...$
    \item Go to \ref{start_simplex}.
\end{enumerate}


\end{document}
\documentclass[9pt]{article}

\usepackage{graphicx} % Required for inserting images
\usepackage{amsmath}
\usepackage{amsfonts}
\usepackage{ctex}
\usepackage{enumitem}
\usepackage{longtable}
\usepackage{makecell} % 换行

% 使用分栏宏包
\usepackage{multicol} 
\setlength{\columnseprule}{0.4pt} % 分割线

% 设置字体
\usepackage{unicode-math}
\setmainfont{Cambria}
\setmathfont{Cambria Math}

% 调整页面布局
\usepackage[a4paper, top=0.7cm, bottom=1cm, left=0.7cm, right=.7cm]{geometry}
\setlength{\footskip}{15pt}

% 设置页脚/页眉
\usepackage{fancyhdr}
\fancyfoot[C]{Copyright By Jingren Zhou | Page \thepage}
\fancyhead[]{}
\pagestyle{fancy}
% 去除线
\renewcommand{\headrulewidth}{0pt}
\renewcommand{\footrulewidth}{0pt}

% 设置 section/subsection 之间的行间距
\usepackage{titlesec}
\titlespacing*{\section}{0pt}{0pt}{0pt}
\titlespacing*{\subsection}{0pt}{0pt}{0pt}

% 调整标题上下间距
\usepackage{titling}
\setlength{\droptitle}{-2.4cm} % 负值表示向上移动

% 设置标题,作者,时间
\title{HAlg Note}
\author{}
\date{}

% 正文
\begin{document}


% 标题
\maketitle
\thispagestyle{fancy}
\vspace{-3.5cm}

% 字体大小
\fontsize{10pt}{11pt}\selectfont
\setlength{\parindent}{8pt}


\section{Basic Knowledge} % Basic Knowledge

\textbf{Def of Matrix}: A mapping from $\{1,...,n\}\times\{1,...,m\}$ to a field $F$ is called a $n\times m$ matrix over $F$.

$\cdot$ The set of all $n\times m$ matrices over $F$ is denoted by $Mat(n\times m;F):=Maps(\{1,...,n\}\times\{1,...,m\},F)$.

$\cdot$ If $n=m$, we sill speak of a \textbf{Square Matrix} and shorten the notation to $Mat(n;F)$.

\textbf{Solution Sets of Inhomogeneous Systems of Linear Equations}: {\small Solution = 特解(Particular Solution) + 通解(Homogeneous solution)}

\textbf{Def of Group $(G,*)$}: A set $G$ with a operator $*$ is a group if: {\small \textbf{Closure}: $\forall g,h\in G$, $g*h\in G$; \textbf{Associativity}: $\forall g,h,k\in G$, $(g*h)*k=g*(h*k)$;}

\hspace{85pt} {\small \textbf{Identity}: $\exists e\in G$, $\forall g\in G$, $e*g=g*e=g$; \textbf{Inverse}: $\forall g\in G$, $\exists g^{-1}\in G$, $g*g^{-1}=g^{-1}*g=e$.}

$\cdot$ \textbf{Properties of Group}: If $G,H$ are groups, then $G\times H$ also.

\textbf{Field $(F)$}: A set $F$ is a field with two operators: \quad {\scriptsize(addition)}$+:F\times F\to F;(\lambda,\mu)\to\lambda+\mu$ \quad {\scriptsize(multiplication)}$\cdot:F\times F\to F;(\lambda,\mu)\to\lambda\mu$ \ \ if:

\quad \quad \quad \quad $(F,+)$ and $(F\setminus\{0_F\}, \ \cdot \ )$ are abelian groups with identity $0_F,1_F$. \quad and \quad $\lambda(\mu+\nu)=\lambda\mu+\lambda\nu$ \quad \quad $e.g. Fields:\mathbb{R},\mathbb{C},\mathbb{Q},\mathbb{Z}/p\mathbb{Z}$


\section{Vector Spaces} % Vector Spaces

\subsection{Vector Spaces | Product of Sets | Vector Subspaces | Power, Union, Intersection of Sets}

\textbf{$F$-Vector Space (V)}: A set $V$ over a field $F$ is a vector space if: \quad $V$ is an abelian group $V=(V,\dotplus)$ and $\forall \ \vec{v},\vec{w}\in V$ \ \ $\lambda,\mu\in F$

\quad \quad \quad \quad a map $F\times V\to V:(\lambda,\vec{v})\to\lambda\vec{v}$ satisfies: \textbf{I}: $\lambda(\vec{v}\dotplus\vec{w})=(\lambda\vec{v})\dotplus(\lambda\vec{w})$ \quad \textbf{II}: $(\lambda+\mu)\vec{v}=(\lambda\vec{v})\dotplus(\mu\vec{v})$

\quad \quad \quad \quad \textbf{III}: $\lambda(\mu\vec{v})=(\lambda\mu)\vec{v}$ \quad \textbf{IV}: $1_F\vec{v}=\vec{v}$ \hspace{30pt} {\scriptsize ps:\textbf{I,II} are \textbf{Distributive Laws}; \textbf{III} is \textbf{Associative Law}. \hspace{30pt} \textbf{Trivial Vector Space}: $V=\vec{0}$}

\begin{enumerate}[itemsep=-2pt, topsep=-2pt]
    \item \textbf{Properties of $F$-Vector Space}: \textbf{a}. $0_F\vec{v}=\vec{0}$ \quad \textbf{b}. $(-1_F)\vec{v}=-\vec{v}$ \quad \textbf{c}. $\lambda\vec{0}=\vec{0}$ \quad \textbf{d}. If $\lambda\vec{v}=\vec{0}$, then $\lambda=0$ or $\vec{v}=\vec{0}$ or both.
    \item If $V,W$ are $F$-vector spaces, then $V\times W$ is also.
\end{enumerate}

\textbf{Component}: An individual entry $x_i$ of an \textbf{n-tuple} $(x_1,...,x_n)$ is called a component.

\textbf{Projections ($pr_i$)}: $pr_i:X_1\times X_2\times\cdots\times X_n \to X_i$ with $(x_1,...,x_n)\mapsto x_i$

\textbf{Vector Subspace (U)}: $U\subseteq V$ is a subspace of $V$ if: \quad \textbf{I}. $\vec{0}\in U$ \quad \textbf{II}. $\forall \vec{u},\vec{v}\in U$, $\forall \lambda\in F$: $\vec{u}+\vec{v}\in U$ and $\lambda\vec{u}\in U$ \quad (or: $\lambda\vec{u}+\mu\vec{v}\in U$)

\begin{enumerate}[itemsep=-2pt, topsep=-2pt]
    \item If $U_1,U_2$ are subspaces of $V$. Then $U_1\cap U_2$ and $U_1+U_2$ are also. \quad {\scriptsize ps: $U_1+U_2:=\{\vec{u}+\vec{v}:\vec{u}\in U_1,\vec{v}\in U_2\}$}
    \item \textbf{Vector Subspace Generated by T ($\langle T \rangle$)}: {\small If $T$ is a subset of a $F$-vector space $V$. \ $\Rightarrow$ \ $\langle T \rangle$ is the smallest subspace of $V$ containing $T$.} \\
    \quad Also, we can get: $\langle T \rangle=span(T):=\{\sum_ic_i\vec{v_i}: if \ T =\{v_1,...,v_i\},c_i\in F\}$ \quad \quad \quad $\forall \vec{v}\in\langle T \rangle,\langle T \cup\{\vec{v}\}\rangle=\langle T \rangle$
    \item \textbf{Generating/Spanning Set}: $V$ is a vector space. If $T\subseteq V$ and $\langle T \rangle=V$. \ $\Rightarrow$ \ $T$ is a generating set of $V$.
    \item \textbf{Finitely Generated}: $\exists$ T finite set, s.t. $V=\langle T\rangle$
\end{enumerate}

\textbf{Power of Set $\mathcal{P}(X)$}: If X is a set, then $\mathcal{P}(X):=\{U:U\subseteq X\}$ {\scriptsize (set of all subsets)} \quad \quad {\small ps: $\mathcal{U}\subseteq\mathcal{P}(X)$ \ $\Rightarrow$ \ $U$ is called a \textbf{system of subsets of $X$}}.

\begin{enumerate}[itemsep=-2pt, topsep=-2pt]
    \item \textbf{Empty System of subsets of X}: Empty System of subsets of X $:=\emptyset\in\mathcal{P}(X)$ \ {\scriptsize (NOT $\{\emptyset\}$)} \quad \quad \quad \star \ $\bigcap\emptyset=X$ \quad and \quad $\bigcup\emptyset=\emptyset$ \ \star
    \item \textbf{Def of Union}: For $\mathcal{U}\subseteq\mathcal{P}(X)$, $\bigcup_{U\in\mathcal{U}}U:=\{x\in X:\exists U\in\mathcal{U} \ s.t. \ x\in U\}$
    \item \textbf{Def of Intersection}: For $\mathcal{U}\subseteq\mathcal{P}(X)$, $\bigcap_{U\in\mathcal{U}}U:=\{x\in X:\forall U\in\mathcal{U},x\in U\}$
\end{enumerate}


\subsection{Linear Independence | Basis | Dimension}

\textbf{Linearly Independent}: $L=\{\vec{v_1},\vec{v_2},...,\vec{v_r}\}$ is linearly independent if: \ $\forall c_1,...,c_r\in F$, $c_1\vec{v_1}+...+c_r\vec{v_r}=\vec{0}$ \ $\Rightarrow$ \ $c_1=...=c_r=0$.

$\cdot$ \textbf{Linearly Dependent}: $L$ is linearly dependent if: $\exists \alpha_1,...,\alpha_r$ not all zero \ s.t. \ $\alpha_1\vec{v_1}+...+\alpha_r\vec{v_r}=\vec{0}$

$\cdot$ Empty set is linearly independent. \quad \quad \quad Every nonzero one-element set is linearly independent.



\subsection{Linear Maps | Rank-Nullity Theorem}



\end{document}
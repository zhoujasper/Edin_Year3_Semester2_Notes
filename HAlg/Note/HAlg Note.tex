\documentclass[9pt]{article}

\usepackage{graphicx} % Required for inserting images
\usepackage{amsmath}
\usepackage{amsfonts}
\usepackage{ctex}
\usepackage{enumitem}
\usepackage{longtable}
\usepackage{multirow}
\usepackage{makecell} % 换行

% 使用分栏宏包
\usepackage{multicol}
\setlength{\columnseprule}{0.4pt} % 分割线

% 设置字体
\usepackage{unicode-math}
\setmainfont{Cambria}
\setmathfont{Cambria Math}

% 调整页面布局
\usepackage[a4paper, top=0.7cm, bottom=1cm, left=0.7cm, right=.7cm]{geometry}
\setlength{\footskip}{15pt}

% 设置页脚/页眉
\usepackage{fancyhdr}
\fancyfoot[C]{Copyright By Jingren Zhou | Page \thepage}
\fancyhead[]{}
\pagestyle{fancy}
% 去除线
\renewcommand{\headrulewidth}{0pt}
\renewcommand{\footrulewidth}{0pt}

% 设置 section/subsection 之间的行间距
\usepackage{titlesec}
\titlespacing*{\section}{0pt}{0pt}{0pt}
\titlespacing*{\subsection}{0pt}{0pt}{0pt}

% 调整标题上下间距
\usepackage{titling}
\setlength{\droptitle}{-2.4cm} % 负值表示向上移动

% 设置标题,作者,时间
\title{HAlg Note}
\author{}
\date{}

% 正文
\begin{document}

\newcommand{\bij}{\stackrel{\sim}{\to}}
\newcommand{\oto}{\hookrightarrow}
\newcommand{\onto}{\twoheadrightarrow}

\newcommand{\np}{{\tiny $^{^{^\ominus}}$}}

\newcommand{\N}{\mathbb{N}}

% 标题
\maketitle
\thispagestyle{fancy}
\vspace{-3.5cm}

% 字体大小
\fontsize{10pt}{11pt}\selectfont
\setlength{\parindent}{8pt}


\section{Basic Knowledge} % Basic Knowledge

\textbf{Lagrange's Theorem}: {\small If $H\subseteq G$ is a subgroup, then $|H|$ divides $|G|$.} \quad \qquad {\scriptsize \textbf{I}: If $G$ is finite, then $g^{|G|}=e$ $\forall g\in G$. \quad \textbf{II}: $o(g) \ | \ |G|$ \quad \textbf{III}: If $|G|=p$ prime, $G$ is cyclic.}

\textbf{\small Complement-wise Operations}: {\scriptsize $\phi:V_1\times V_2\to V_1\oplus V_2$ by \textbf{I}:$(\vec{v_1},\vec{u_1})+(\vec{v_2},\vec{u_2}):=(\vec{v_1}+\vec{v_1},\vec{u_1}+\vec{u_2}), \ \lambda(\vec{v},\vec{u}):=(\lambda\vec{v},\lambda\vec{u})$ {\tiny (ps:$V_1,V_2$通过$\phi$定义的map所形成的vector space记作$V_1\oplus V_2$)}}

\textbf{External Direct Sum}: {\small 一个"代数结构"(Vector Space), 定义为set是$V_1\oplus\cdots\oplus V_n:=V_1\times\cdots\times V_n$且有一组运算法则component-wise operations}

\textbf{Projections}: $pr_i:X_1\times\cdots\times X_n\to X_i$ by $(x_1,...,x_n)\mapsto x_i$ \quad \textbf{Canonical Injections}: $in_i:X_i\to X_1\times\cdots\times X_n$ by $x\mapsto(0,...,0,x,0,...,0)$


\section{Summary}
\vspace{-10pt}
{\scriptsize
\begin{longtable}{l@{\hskip 2pt}||l|l|l|l}
    \textbf{\tiny Name}  & \textbf{Group $(G,*)$}                                                        & \textbf{Ring $(R,+,\ \cdot \ )$}                                                   & \textbf{Vector Space $(F-V)$}                                                                                              & \textbf{Module $(R-M)$}                                                                   \\
    \hline
    \hline
    \textbf{\tiny Def}   & \textbf{Closure}: $g*h\in G$ \hfill{\tiny $\forall g,h,k\in G$}               & $(R,+)$ is \textit{abelian group} with $0_R$ \hfil{\tiny $\forall a,b,c\in R$}     & $(V,\dotplus)$ is \textit{abelian group} \hfill{\tiny $\forall \ \vec{v},\vec{w}\in V$}                                    & $(M,\dotplus)$ is \textit{abelian group} \hfill{\tiny $\forall \ m_1,m_2\in M$}           \\
                         & \textbf{Associativity}: $(g*h)*k=g*(h*k)$                                     & $(R,\cdot)$ is \textbf{monoid} with $1_R$ {\tiny (monoid is closure)}              & {\tiny $\exists$ map $F\times V\to V:(\lambda,\vec{v})\to\lambda\vec{v}$} \hfill{\tiny $\forall \ \lambda,\mu\in F$}       & {\tiny $\exists$ map $R\times M\to M:(r,m)\to rm$} \hfill{\tiny $\forall \ r_1,r_2\in R$} \\
                         & \textbf{Identity}: $\exists e\in G$, $e*g=g*e=g$                              & \qquad {\tiny i.e. \textbf{Associativity}: , $(a\cdot b)\cdot c=a\cdot(b\cdot c)$} & \textbf{I}: $\lambda(\vec{v}\dotplus\vec{w})=(\lambda\vec{v})\dotplus(\lambda\vec{w})$                                     & \textbf{I}: $r(m_1\dotplus m_2)=(\lambda m_1)\dotplus(\lambda m_2)$                       \\
                         & \textbf{Inverse}: $\exists g^{-1}\in G$, $g*g^{-1}=g^{-1}*g=e$                & \qquad \quad \ \ {\tiny \textbf{Identity}: $1_R\cdot a=a\cdot 1_R=a$}              & \textbf{II}: $(\lambda+\mu)\vec{v}=(\lambda\vec{v})\dotplus(\mu\vec{v})$                                                   & \textbf{II}: $(r_1+r_2)m_1=(r_1m_1)\dotplus(r_1m_1)$                                      \\
                         &                                                                               & \textbf{Distributive}: $a\cdot(b+c)=a\cdot b+a\cdot c$                             & \textbf{III}: $\lambda(\mu\vec{v})=(\lambda\mu)\vec{v}$                                                                    & \textbf{III}: $r_1(r_2m_1)=(r_1r_2)m_1$                                                   \\
                         &                                                                               & \qquad \qquad \qquad $(b+c)\cdot a=b\cdot a+c\cdot a$                              & \textbf{IV}: $1_F\vec{v}=\vec{v}$                                                                                          & \textbf{IV}: $1_Rm_1=m_1$                                                                 \\
    \hline
    \textbf{\tiny Prop}  & \textbf{I}: $(gh)^{-1}=h^{-1}g^{-1}$                                          & \textbf{I}. $0\cdot a=a\cdot 0=0$ \hfill{\tiny $\forall a,b\in R$}                 & \textbf{I}. $0\vec{v}=0$ and $\vec{0}\lambda=\vec{0}$ \hfill{\tiny $\forall \vec{v}\in V,\lambda\in F$}                    & \textbf{I}. $0_Rm=0_M$ ; $r0_M=0_M$ \hfill{\tiny $\forall r\in R,m\in M$}                 \\
                         &                                                                               & \textbf{II}. $(-a)\cdot b=a\cdot(-b)=-(a\cdot b)$                                  & \textbf{II}. $(-1)\vec{v}=-\vec{v}$                                                                                        & \textbf{II}. $(-r)m=r(-m)=-(rm)$                                                          \\
                         &                                                                               & {\tiny \textbf{Commutative Ring}: add $\forall a,b\in R$, $ab=ba$}                 & \textbf{III}. $\lambda\vec{v}=\vec{0}$ $\Leftrightarrow$ $\lambda=0$ or $\vec{v}=\vec{0}$ $\star$                          &                                                                                           \\
    \hline
    \textbf{\tiny Remark}& $G,H$ groups $\Rightarrow$ $G\times H$ also.                                  & For ring $R$ [$1_R=0_R$ $\Leftrightarrow$ $R=\{0\}$]                               &                                                                                                                            &                                                                                           \\
    \hline
    \textbf{\tiny e.g.}  & Cyclic group; $GL_n$ ; $D_n$ ; $\mathbb{Z}$                                   & $Mat(n,F)$ ; $R[X]$ ; $\mathbb{Z}/m\mathbb{Z}$ ; $\mathbb{Z}$                      & $\mathbb{R}[x]_{<n}$ ; $Mat(n,F)$ ; $Hom(V,W)$                                                                             & {\tiny $R=\mathbb{Z}$ Abelian Group; $R=F$ Vector Space}                                  \\
    \hline
    \hline
    \textbf{\tiny Sub}     & \textbf{Subgroup ($H$)}: \hfill{\tiny $\forall h_1,h_2\in H$}               & \textbf{Subring ($R'$)}: \hfill{\tiny $\forall a,b\in R'$}                         & \textbf{Subspace ($U$)}: \hfill{\tiny $\forall \ \vec{v},\vec{u}\in U,\lambda,\mu\in F$}                                   & \textbf{Submodule ($M'$)}: \hfill{\tiny $\forall m_1,m_2\in M'$}                          \\
    \textbf{\tiny objects} & \textbf{I}: $H\ne\emptyset$;                                                & \textbf{I}. $1_R\in R'$                                                            & \textbf{I}. $\vec{0}\in U$                                                                                                 & \textbf{I}. $0_M\in M'$ \hfill{\tiny $\forall r_1,r_2\in R$}                              \\
                           & \textbf{II}: $h_1*h_2\in H$;                                                & \textbf{II}. $a-b\in R'$                                                           & \textbf{II}. $\vec{u}+\vec{v}\in U$ and $\lambda\vec{u}\in U$                                                              & \textbf{II}. $m_1-m_2\in M'$ and $r_1m_1\in M'$                                           \\
                           & \textbf{III}: $h_1^{-1}\in H$.                                              & \textbf{III}. $ab\in R'$                                                           & \qquad \quad (or: $\lambda\vec{u}+\mu\vec{v}\in U$)                                                                        & \qquad \quad (or: $r_1m_1-r_2m_2\in M'$)                                                  \\
    \hline
    \textbf{\tiny Create}  & $H,K$ subgroups $\Rightarrow$ $H\cap K$ also.                               & $R,S$ subring $\Rightarrow$ $R\cap S$ also.                                        & $V,W$ subspaces $\Rightarrow$ $V\cap W,V+W$ also.                                                                          & $M,N$ submodules $\Rightarrow$ $M\cap N,M+N$ also.\\
    \hline
    \hline
    \hline
    \textbf{\tiny Generate}& \textbf{Generated Group $\langle T\rangle$}:                                & \textbf{Generated Ideal $_R\langle T\rangle$}: \hfill \textbf{\tiny $R$ is \textit{commutative ring}} & \textbf{Generated subspaces $\langle T\rangle$}:                                                                & \textbf{Generated submodules} $_R\langle T\rangle$                                \\
    \textbf{\tiny objects} & $\langle T\rangle:=\{g_1^{a_1}...g_k^{a_k}|k\in\N,g_i\in T,a_i\in\N\}$      & $_R\langle T\rangle:=\{\sum_{i=1}^{n}r_it_i:n\in\mathbb{N},r_i\in R,t_i\in T\}$                       & {\tiny $\langle T\rangle:=\{\alpha_1\vec{v_1}+\cdots+\alpha_n\vec{v_n}:\alpha_i\in F,\vec{v_i}\in T,n\in\N\}$}  & {\tiny $\langle T\rangle:=\{r_1t_1+\cdots+rt_n:r_i\in R,t_i\in T,n\in\N\}$}       \\
    \hline
    \textbf{\tiny Special} & \textbf{Cyclic Group}: $\langle g\rangle=\{g^k|k\in\mathbb{Z}\}$            & \textbf{Principal Ideal}: $_R\langle a\rangle$ \quad i.e. $aR$                                        & $\langle\emptyset\rangle:=\{\vec{0}\}$                                                                          & \textbf{Cyclic submodule}: If $M=_R\langle t\rangle$                              \\
    \hline
    \textbf{\tiny Prop}    & \multicolumn{4}{c}{$\langle T\rangle$ is the smallest the \{generated things\} containing $T$. \qquad \qquad {\tiny ps: 默认 $^2T\subseteq R$ \quad $^4 T\subseteq M$}} \\
    \hline
    \hline
    \hline
    \textbf{\tiny Homo}    & \textbf{\tiny Homomorphism}:$\phi:G\to H$\hfill{\tiny $\forall g_1,g_2\in G$}& $f:R\to S$ hom: \hfill{\tiny $\forall a,b\in R$}                                  & $f:V\to W$ \hfill{\tiny $\forall \vec{v}_1,\vec{v}_2\in V, \lambda\in F$}                                                          & \textbf{R-Hom}: $f:M\to N$\hfill{\tiny $\forall a,b\in M$, $r\in R$}              \\
                           & \textbf{I}. $\phi(g_1*g_2)=\phi(g_1)*\phi(g_2)$                              & \textbf{I}. $f(a+b)=f(a) + f(b)$                                                  & \textbf{I}. $f(\vec{v}_1+\vec{v}_2)=f(\vec{v}_1)+f(\vec{v}_2)$                                                                     & \textbf{I}. $f(a+b)=f(a)+f(b)$                                                    \\
                           &                                                                              & \textbf{II}.$f(ab)=f(a)f(b)$                                                      & \textbf{II}. $f(\lambda\vec{v}_1)=\lambda f(\vec{v}_1)$                                                                            & \textbf{II}. $f(ra)=rf(a)$                                                        \\
    \hline
    \textbf{\tiny Prop A}  & \textbf{I}: $\phi(e_G)=e_H$                                                  & \textbf{I}. $f(0_R)=0_S$ \qquad {\tiny $f(1_R)=1_S$ NOT need}                     & \textbf{I}. $f(\vec{0})=\vec{0}$                                                                                                   & \textbf{I}. $f(0_M)=0_N$ \qquad {\tiny $f(1_R)=1_S$ NOT need}                     \\
                           & \textbf{II}: $\phi(g^{-1})=\phi(g)^{-1}$                                     & \textbf{II}. $f(x-y)=f(x)-f(y)$                                                   & \textbf{II}. $f(\lambda\vec{v}+\mu\vec{u})=\lambda f(\vec{v})+\mu f(\vec{u})$                                                      & \textbf{II}. $f(a-b)=f(a)-f(b)$                                                   \\
                           &                                                                              & \textbf{III}. $f(a^n)=(f(a))^n$ \quad $f(mx)=mf(x)$                               & \textbf{III}. $f\circ g$ is linear map.                                                                                            &                                                                                   \\
                           & \textbf{III}. $\phi$ is 1-1 $\Leftrightarrow$ $\ker\phi=\{e_G\}$             & \textbf{Iv}. $f$ is 1-1 $\Leftrightarrow$ $\ker f=\{0_R\}$                        & \textbf{IV}. $f$ is 1-1 iff $\ker f=\{\vec{0}\}$                                                                                   & \textbf{III}. $f$ is 1-1 iff $\ker f=\{0\}$                                       \\
    \hline
    \textbf{\tiny Ker/Im}  & \textbf{I}. $Im(\phi)$ subgroup \quad $\ker(\phi)\lhd G$ normal.             & \textbf{I}. $Im(f)$ subring. \quad $\ker(f)\trianglelefteq R$ ideal .             & \textbf{I}. $\ker(f)$ ; $Im(f)$ are subspaces.                                                                                     & \textbf{I}.$\ker f,Im f$ are submodules.                                          \\
                           & \textbf{II}. $K\subseteq G$ is subgroup \ $\Rightarrow$ \ $\phi(K)\subseteq H$ also. & \textbf{II}. $R'\subseteq R$ is subring \ $\Rightarrow$ \ $f(R')$ also.   & \textbf{II}. Rank-Nullity Theorem...                                                                                               &                                                                                   \\
                           & \textbf{III}. $Ker(\phi)$ subgroup.                                          &                                                                                   &                                                                                                                                    &                                                                                   \\
    \hline
    \textbf{\tiny Remark}  & \multicolumn{4}{c}{\textbf{Isomorphism}: $=$ LM \& Bij. \ \ \textbf{Endomorphism(End)}: $=$ LM \& $V=W$. \ \ \textbf{Automorphism(Aut)}: $=$ Iso \& $V=W$ \ \ \textbf{Monomorphism}: $=$ LM \& 1-1. \ \ \textbf{Epimorphism}: $=$ LM \& onto.}

\end{longtable}
}
\vspace{-8pt}

\textbf{Normal $(H\lhd G)$}: {\small $H\subseteq G$ is normal if: $\forall g\in G$, $gH=Hg$} 

\qquad \textbf{Property}: \textbf{I}: $Ker\phi\lhd G$ \quad \textbf{II}: $\phi$ is 1-1 $\Rightarrow$ $G\cong im\phi$

\textbf{Ideal $(I\trianglelefteq R)$}: A subset $I\subseteq R$ (ring) is an ideal if: \quad \textbf{I}. $I\ne\emptyset$ \quad \textbf{II}. $\forall a,b\in I$, $a-b\in I$ \quad \textbf{III}. $\forall i\in I,\forall r\in R$, $ri,ir\in I$ \quad {\scriptsize e.g.$m\mathbb{Z}$}

\qquad \textbf{Property}: If $I,J$ are \textit{ideals} of $R$. Then \ \ $I+J$ \ ; \ $I\cap J$ \ \ are also ideals.

\textbf{Field $(F)$}: {\small A set $F$ is a field with two operators: \quad {\scriptsize(addition)}$+:F\times F\to F;(\lambda,\mu)\to\lambda+\mu$ \quad {\scriptsize(multiplication)}$\cdot:F\times F\to F;(\lambda,\mu)\to\lambda\mu$ \ \ if:}

\quad \quad \quad \quad {\small $(F,+)$ and $(F\setminus\{0_F\}, \ \cdot \ )$ are abelian groups with identity $0_F,1_F$. \quad and \quad $\lambda(\mu+\nu)=\lambda\mu+\lambda\nu$ \quad \quad $e.g. Fields:\mathbb{R},\mathbb{C},\mathbb{Q},\mathbb{Z}/p\mathbb{Z}=\mathbb{F}_p$}

\qquad \qquad  \textbf{Field}: For a ring $R$: Commutative ring + $R$ has multiplicative inverse = Field.


\section{Vector Spaces/Subspaces | Generating Set | Linear Independent | Basis} % Vector Spaces | Subspaces | Generating Set | Linear Independent

\textbf{Linearly Independent}: $L=\{\vec{v_1},\vec{v_2},...,\vec{v_r}\}$ is linearly independent if: \ $\forall c_1,...,c_r\in F$, $c_1\vec{v_1}+\cdots+c_r\vec{v_r}=\vec{0}$ \ $\Rightarrow$ \ $c_1=\cdots =c_r=0$.

$\cdot$ \textbf{Connect to Matrix}: {\footnotesize Let $L=\{\vec{v_1},...,\vec{v_n}\}$, $L$ is LI of $V$. Let $A=[\vec{v_1},...,\vec{v_n}]$ $\Rightarrow$ $\forall\vec{x}\in F^n$, $A\vec{x}=0$ $(or \ \vec{0})$ $\Rightarrow$ $\vec{x}=0 (or \ \vec{0})$ (i.e. linear map $\phi$:$\vec{x}\mapsto A\vec{x}$ is injective)}

\textbf{Basis \& Dimension}: If $V$ is finitely generated. $\Rightarrow$ $\exists$ subset $B\subseteq V$ which is both LI and GS. ($B$ is basis) \quad \textbf{Dim}: $\dim V:=|B|$

$\cdot$ \textbf{Connect to Matrix}: {\footnotesize Let $B=\{\vec{v_1},...,\vec{v_n}\}$ is basis of $V$. Let $A=[\vec{v_1},...,\vec{v_n}]$ $\Rightarrow$ $\forall\vec{x}=(x_1,...,x_n)^T$ s.t. $\phi:\vec{x}\mapsto A\vec{x}$ is 1-1 \& onto (Bijection)}

\textbf{Relation|GS,LI,Basis,dim}: {\small Let $V$ be vector space. $L$ is linearly independent set, $E$ is generating set, $B$ is basis set.}

\begin{enumerate}[itemsep=-2pt, topsep=-2pt]
    \item \textbf{GS|LI}: $|L|\leq|E|$ {\scriptsize (can get: $\dim$ unique)} \quad \textbf{LI$\to$Basis}: If $V$ finite generate $\Rightarrow$ $\forall L$ can extend to a basis. {\scriptsize If $L=\emptyset$, prove $\exists B$} \quad \quad $ker f\cap im f=\{0\}$
    \item \textbf{Basis|max,min}: $B$ $\Leftrightarrow$ $B$ is minimal GS $(E)$ $\Leftrightarrow$ $B$ is maximal LI $(L)$. \quad \textbf{Uniqueness|Basis}: {\small 每个元素都可以由basis唯一表示.}
    \item \textbf{Proper Subspaces}: {\small If $U\subset V$ is proper subspace, then $\dim U<\dim V$. \quad $\Rightarrow$ \quad If $U\subseteq V$ is subspace and $\dim U=\dim V$, then $U=V$.}
    \item \textbf{Dimension Theorem}: If $U,W\subseteq V$ are subspaces of $V$, then $\dim(U+W)=\dim U+\dim W-\dim(U\cap W)$
\end{enumerate}

\textbf{Complementary}: $U,W\subseteq V$,$U,V$ subspaces are complementary ($V=U\oplus W$) if: \quad $\exists\phi:U\times W\to V$ by $(\vec{u},\vec{w})\mapsto\vec{u}+\vec{w}$

\quad \quad \quad \quad \quad \quad \quad \quad i.e. $\forall\vec{v}\in V$, we have unique $\vec{u}\in U,\vec{w}\in W$ s.t. $\vec{v}=\vec{u}+\vec{w}$. \quad {\footnotesize ps: It's a linear map.}


\section{Linear Mapping | Rank-Nullity| Matrices | Change of Basis \quad {\tiny ps: 默认$V,W$ \ $F$-Vector Spaces.}}

\subsection{Linear Mapping | Rank-Nullity} % Linear Mapping | Rank-Nullity

\textbf{Property of Linear Map}: Let $f,g\in Hom$

\begin{enumerate}[itemsep=-2pt, topsep=-2pt]
    \item \textbf{Determined}: $f$ is determined by $f(\vec{b_i}),\vec{b_i}\in \mathcal{B}_{basis}$ {\footnotesize (\star \ \textbf{i.e.} $f(\sum_{i}\lambda_i\vec{v_i}):=\sum_{i}\lambda_if(\vec{v_i})$)}
    \item \textbf{Classification of Vector Spaces}: $\dim V=n$ \ $\Leftrightarrow$ \ $f:F^n\bij V$ by $f(\lambda_1,...,\lambda_n)\mapsto\sum_{i=1}^{n}\lambda_i\vec{v_i}$ is isomorphism.
    \item \textbf{Left/Right Inverse}: $f$ is 1-1 $\Rightarrow$ $\exists$ left inverse $g$ s.t. $g\circ f=id$ {\scriptsize 考虑direct sum} \quad \quad $f$ is onto $\Rightarrow$ $\exists$ right inverse $g$ s.t. $f\circ g=id$
    \item \np\textbf{More of Left/Right Inverse}: $f\circ g=id$ $\Rightarrow$ $g$ is 1-1 and $f$ is onto. \quad {\scriptsize 使用kernel=0来证明}
\end{enumerate}

\textbf{Rank-Nullity Theorem}: For linear map $f:V\to W$, $\dim V=\dim(\ker f)+\dim(Im f)$ \quad \quad \quad {\footnotesize Following are properties:}

\begin{enumerate}[itemsep=-2pt, topsep=-2pt]
    \item \textbf{Injection}: $f$ is 1-1 $\Rightarrow$ $\dim V \leq \dim W$ \quad \textbf{Surjection}: $f$ is onto $\Rightarrow$ $\dim V \geq \dim W$ \quad Moreover, $\dim W=\dim imf$ iff $f$ is onto.
    \item \textbf{Same Dimension}: {\small $f$ is isomorphism $\Rightarrow$ $\dim V=\dim W$} \quad \quad \quad \textbf{Matrix}: $\forall M$, column rank $c(M)$ = row rank $r(M)$.
    \item \textbf{Relation}: {\small If $V,W$ finite generate, and $\dim V=\dim W$, \quad Then: $f$ is isomorphism $\Leftrightarrow$ $f$ is 1-1 $\Leftrightarrow$ $f$ is onto.}
\end{enumerate}


\subsection{Matrices | Change of Basis | Similar Matrices | Trace} % Matrices | Change of Basis | Similar Matrices | Trace

\textbf{Matrix}: For $A_{n\times m},B_{m\times p}$, $AB_{n\times p}:=(AB)_{ij}=\sum_{k=1}^{m}a_{ik}b_{kj}$ \quad \textbf{Transpose}: $A^T_{m\times n}:=(A^T)_{ij}=a_{ji}$

\textbf{Invertible Matrices}: {\small $A$ is invertible if $\exists B,C$ s.t. $BA=I$ and $AC=I$ \quad || \quad $\exists B, \ BA=I$ $\Leftrightarrow$ $\exists C,AC=I$ $\Leftrightarrow \exists A^{-1}$} \quad {\small $_{\mathcal{B}}[f^{-1}]_{\mathcal{A}}=_{\mathcal{A}}[f]^{-1}_{\mathcal{B}}$}

\textbf{Representing matrix of linear map $_{\mathcal{B}}[f]_{\mathcal{A}}$}: $f:V\to W$ be linear map, $\mathcal{A}=\{\vec{v_1},...,\vec{v_n}\}$ is basis of $V$, $\mathcal{B}=\{\vec{w_1},...,\vec{w_m}\}$ is basis of $W$.

\begin{enumerate}[itemsep=-2pt, topsep=-2pt]
    \item $_{\mathcal{B}}[f]_{\mathcal{A}}:=A$ {\small (matrix)} where $f(\vec{v}_{i\in\mathcal{A}})=\sum_{j\in\mathcal{B}}A_{ji}\vec{w}_j$ \quad \quad $\exists M^{\mathcal{A}}_{\mathcal{B}}:Hom_{F}(V,W)\bij Mat(n\times m;F)$
    \item If $\vec{v}\in V$, then $_{\mathcal{A}}[\vec{v}]:=\mathbf{b}$ {\small (vector)} where $\vec{v}=\sum_{i\in\mathcal{A}}\mathbf{b}_i\vec{v_i}$
    \item \textbf{Theorems}: $[f\circ g]=[f]\circ[g]$ \quad \quad $_{\mathcal{C}}[f\circ g]_{\mathcal{A}}=_{\mathcal{C}}[f]_{\mathcal{B}}\circ_{\mathcal{B}}[g]_{\mathcal{A}}$ \quad \quad $_{\mathcal{B}}[f(\vec{v})]=_{\mathcal{B}}[f]_{\mathcal{A}}\circ_{\mathcal{A}}[\vec{v}]$ \quad \quad $_{\mathcal{A}}[f]_{\mathcal{A}}=I\Leftrightarrow f=id$
    \item \textbf{Change of Basis}: {\small Define \textit{Change of Basis Matrix}$:=$$_{\mathcal{A}}[id_V]_{\mathcal{B}}$ \quad \quad {\footnotesize $_{\mathcal{B}'}[f]_{\mathcal{A}'}=_{\mathcal{B}'}[id_W]_{\mathcal{B}}\circ$$_{\mathcal{B}}[f]_{\mathcal{A}}\circ$$_{\mathcal{A}}[id_V]_{\mathcal{A}'}$ \quad \quad $_{\mathcal{A}'}[f]_{\mathcal{A'}}=_{\mathcal{A}}[id_V]_{\mathcal{A'}}^{-1}\circ$$_{\mathcal{A}}[f]_{\mathcal{A}}\circ$$_{\mathcal{A}}[id_V]_{\mathcal{A}'}$}}
\end{enumerate}

\textbf{Elementary Matrix}: $I+\lambda E_{ij}$ {\scriptsize (cannot $I-E_{ii}$)} {\footnotesize 就是初等矩阵,左乘代表$j$行乘$\lambda$倍加到第$i$行,右乘代表$j$列乘$\lambda$倍加到第$i$列} \quad $\Rightarrow$ \quad Invertible!

\begin{enumerate}[itemsep=-2pt, topsep=-2pt]
    \item {\footnotesize 交换$i,j$列/行}: $P_{ij}=diag(1,...,1,-1,1,...,1)(I+E_{ij})(I-E_{ji})(I+E_{ij})$ where $-1$ in $j$th place.
    \vspace{-2pt}
    \item \textbf{Row Echelon Form|Smith Normal Form}: $\stackrel{\sim}{A}:REF$ {\footnotesize 通过左乘初等矩阵可以实现} \quad \quad $\stackrel{\approx}{A}:S(n,m,r)$ {\footnotesize 通过$\stackrel{\sim}{A}$右乘初等矩阵可以实现}
\end{enumerate}

\textbf{Smith Normal Form}: {\small $\forall A$, $\exists \ \text{invertible} \ P,Q$ s.t. $PAQ=S(n,m,r):=$ {\footnotesize $n\times m$的矩阵,对角线前$r$个是1,后面0}. \quad \textbf{Lemma}: $r=r(A)=c(A)$}

$\cdot$ Every linear map $f:V\to W$ can be representing by $_{\mathcal{B}}[f]_{\mathcal{A}}=S(n,m,r)$ for some basis $\mathcal{A},\mathcal{B}$ of $V,W$.

\textbf{Similar Matrices}: $N=T^{-1}MT$ $\Leftrightarrow$ $M,N$ are similar. \quad {\small \textit{Special Case}: If $N=$$_{\mathcal{B}}[f]_{\mathcal{B}}$, $M=$$_{\mathcal{A}}[f]_{\mathcal{A}}$, then $N=T^{-1}MT$. where $T=$$_{\mathcal{A}}[id_V]_{\mathcal{B}}$}

\begin{enumerate}[itemsep=-2pt, topsep=-2pt]
    \item If $A\sim B$ iff $A$ is similar to $B$, then $\sim$ is an equivalence relation. \quad \quad $_{\mathcal{A}'}[f]_{\mathcal{A'}}\sim_{\mathcal{A}}[f]_{\mathcal{A}}$
    \item If $\mathcal{B}=\{p(\vec{v_1}),...,p(\vec{v_n})\}$ and $\mathcal{A}=\{\vec{v_1},...,\vec{v_n}\}$ where $p:V\bij V$. Then $_{\mathcal{A}}[id_V]_{\mathcal{B}}=_{\mathcal{A}}[p]_{\mathcal{A}}$
    \item If $V$ is a vector space over $F$, [$A,B$ are \textit{similar} matrices. \ \ $\Leftrightarrow$ \ \ $A=_{\mathcal{A}}[f]_{\mathcal{A}},B=_{\mathcal{B}}[f]_{\mathcal{B}}$ for some basis $\mathcal{A},\mathcal{B}$; $f:V\to V$]
    \item {\small Set of \textit{Endomorphism} is in a bijection correspondence with the equivalence class of matrices under $\sim$. } {\scriptsize \textbf{一个自同态End就对应一个相似矩阵的等价类}}
\end{enumerate}

\textbf{Trace}: $tr(A):=\sum_{i}a_{ii}$ and $tr(f):=tr(_{\mathcal{A}}[f]_{\mathcal{A}})$ \ $\big|$ \ {\small $tr(AB)=tr(BA)$ \quad $tr(\lambda A+\mu B)=\lambda tr(A)+\mu tr(B)$ \quad $tr(N)=tr(M)$ if $M,N$ similar.}


\section{Rings | Polynomials | Ideals | Subrings} % Rings | Polynomials | Ideals | Subrings

\subsection{Rings | Polynomial Rings} % Rings | Polynomial Rings

\textbf{2nd Def of Ring Homomorphism}: {\small $f$ is ring homomorphism if: \quad 1. $f:(R,+)\to (S,+)$ is group homomorphism \ \ and \ \ 2. $f(xy)=f(x)f(y)$.}

\textbf{Unit}: $a\in R$ is unit if it's \textit{Invertible}. \quad i.e. $\exists a^{-1}\in R$ s.t. $aa^{-1}=a^{-1} a=1_R$ \qquad \textbf{Group of Unit} $(R^{\times},\ \cdot \ ):=\{a\in R:a$ is unit$\}$

$\cdot$ \textbf{Lemma}: If $^1f:R\to S$ homo, $^2f(1_R)=1_S$, $^3x$ is \textit{unit} of $R$. \ \ $\Rightarrow$ \ \ $^1f(x)$ is \textit{unit} of $S$. \quad $^2f|_{R^\times}:R^\times\to S^\times$ is \textit{group homomorphism}.

\textbf{Zero-divisors}: $a\in R$ is \textit{zero-divisor} if $\exists b\in R,b\ne0$ s.t. $ab=0$ or $ba=0$ \qquad \textit{Field has no zero-divisors.}  \quad {\scriptsize $\cdot$ e.g. $\mathbb{Z}^{\times}=\{-1,1\}$ \ ; \ $1_R$ is a unit.}

\textbf{Integral Domain}: A \textit{commutative} ring $R$ is an integral domain if it has no zero-divisors. \qquad \quad {\scriptsize e.g. $\mathbb{Z}/p\mathbb{Z},\mathbb{R},\mathbb{Q},\mathbb{Z},...$}

\textbf{Properties of Integral Domain}: $\forall a,b\in R$. \quad \textbf{I}. $ab=0$ $\Rightarrow$ $a=0$ or $b=0$ \quad \textbf{II}. $a,b\ne0$ $\Rightarrow$ $ab\ne0$ \qquad \textbf{III}. $ac=bc,a\ne0$ $\Rightarrow$ $b=c$

$\cdot$ \textit{Field is Integral Domain} \qquad \textbf{Every finite integral domain is a field} \qquad $\mathbb{Z}/p\mathbb{Z}$ is field iff $p$ is prime. \qquad {\scriptsize e.g.(integral domain) $\mathbb{Z};\mathbb{Z}/p\mathbb{Z}$}

\textbf{Polynomial Ring $R[X]$}: $R[X]:=\{a_nX^n+\cdots+a_1X+a_0:a_i\in R,n\in\mathbb{N}\}$ where $X$ is \textbf{indeterminate} \ $\Leftarrow$ \ $X\notin R$ and $\forall x\in R,Xa=aX$

\begin{enumerate}[itemsep=-2pt, topsep=-2pt]
    \item \textbf{Degree}: $\deg(P):=\max\{n\in\mathbb{N}:a_n\ne0\}$ \quad \textbf{Leading Coefficient}: $a_n$ \quad \textbf{Monic}: $a_n=1$ \quad \quad \quad {\scriptsize ps: Polynomial NOT a function}
    \item \textbf{Lemma}: {\small $^1$ $R$ \textit{integral domain|no zero-divisors} $\Rightarrow$ $R[X]$ also. \qquad $^2$ $R$ \textit{integral domain or no zero-divisor} $\Rightarrow\deg(PQ)=\deg(P)+\deg(P)$}
    \item \textbf{Division and Remainder}: If $R$ is \textit{integral domain} and $P,Q\in R[X]$,{\scriptsize $Q$ \textbf{monic}} $\exists!$ $A,B\in R[X]$ s.t. $P=AQ+B$ and $\deg(B)<\deg(Q)$
    \item \textbf{Function | Factorize}: If $R$ is \textit{commutative ring} \ $\Rightarrow$ \ $^1$ $R[X]\to Maps(R,R)$ {\scriptsize (可以视作函数)} \ $^2$ $\lambda\in R$ is root of $P$ $\Leftrightarrow$ $(X-\lambda) \ | \ P(X)$
    \item \textbf{Roots}: If $R$ is \textit{Integral domain}: $P$ has at most $\deg(P)$ roots.
\end{enumerate}

\textbf{Algebraically Closed}: $R=F$ field is \textit{algebraically closed} if every non-constant polynomial has a root in $F$. \qquad \qquad e.g. $\mathbb{C}$

$\cdot$ \textbf{Decomposes}: If $F$ field is \textit{algebraically closed} $\Rightarrow$ $P$ decomposes into: $P(X)=a(X-\lambda_1)\cdots(X-\lambda_n),a\in F^\times$ {\scriptsize i.e. $a\ne0$}


\subsection{Equivalence Relation}

\textbf{Equivalence Relation}: A relation $R$ on a set $X$ is a subset $R\subseteq X\times X$. If $(x,y)\in R$, we write $xRy$, {\footnotesize if $R$ is Equivalence Relation, then:}

\quad \textbf{Reflexive}: $xRx$ {\scriptsize ($x\sim x$)} \quad \textbf{Symmetric}: $xRy\Rightarrow y Rx$ {\scriptsize ($x\sim y \Rightarrow y\sim x$)} \quad \textbf{Transitive}: $xRy,yRz\Rightarrow xRz$ {\scriptsize ($x\sim y,y\sim z\Rightarrow x\sim z$)}

\textbf{Partial Order}: A relation $R$ on a set $X$, $xRy$. {\footnotesize If $R$ is partial order, then:}

\quad \textbf{Reflexive}: $xRx$ {\scriptsize ($x\sim x$)} \quad \textbf{Anti-symmetric}: $xRy,yRx\Rightarrow x = y$ {\scriptsize ($x\sim y,y\sim x\Rightarrow x=y$)} \quad \textbf{Transitive}: $xRy,yRz\Rightarrow xRz$ {\scriptsize ($x\sim y,y\sim z\Rightarrow x\sim z$)}

\textbf{Property of Equivalence Relation}: If $R$ $(\sim)$ is equivalence relation on $X$.

\begin{enumerate}[itemsep=-2pt, topsep=-2pt]
    \item $\sim$ Define the \textbf{equivalence classes} of $x\in X$ as $E(x):=\{y\in X:x\sim y\}$
    \item $\sim$ \textbf{Partition} $X$ into disjoint subsets $X=\bigcup_{i}X_i$, $X_i$ is equivalence class of $x\in X$.
    \item $x\sim y$ \quad $\Leftrightarrow$ \quad $E(x)=E(y)$ \quad $\Leftrightarrow$ \quad $E(x)\cap E(y)\ne\emptyset$.
\end{enumerate}

\textbf{Set of Equivalence Classes $(X/\sim)$}: $(X/\sim):=\{E(x):x\in X\}$ \qquad \textbf{Canonical Projection}: $can:X\to (X/\sim)$ by $x\mapsto E(x)$

\quad \textbf{System of Representatives}: $Z\subseteq X$ is a system of representatives if {\footnotesize 每个等价类都恰好有一个元素代表在$Z$中}

\quad \textbf{Examples}: {\tiny $^1$ If $V$ $F-$vector space, $W$ subspace. Then $V/W$ is \textbf{quotient vector space}. \quad $^2$ If $G$ group, $H$ normal. Then $G/H$ is \textbf{quotient group}. \quad $^3$ If $R$ ring, $I$ ideal. Then $R/I$ is \textbf{quotient ring}.}

\textbf{Universal Property of the set of Equivalence Classes}: If $f:X\to Z$ is a map s.t. $x\sim y$ $\Leftrightarrow$ $f(x)=f(y)$. {\tiny ($\sim$ is Equivalence relation) \textbf{Important}}

\quad Then, $\exists!$ map $\overline{f}:(X/\sim)\to Z$ s.t. $f=\overline{f}\circ can$ \quad with \quad $\overline{f}(E(x))=f(x)$ is \textit{well-defined}. \qquad Further more, $\overline{f}:(X/\sim)\bij Im(f)$

\quad ps: Often, if we want to prove $g:(X/\sim)\to Z$ is well-defined, we need to prove $x\sim y$ $\Leftrightarrow$ $g(x)=g(y)$ holds.


\subsection{Factor Ring | First Isomorphism Theorem} % Factor Ring | First Isomorphism Theorem

\textbf{Coset of Ideal}: Let $I$ be an ideal of $R$. Then $a+I$ is a coset of $I$. \quad The $\sim$ is defined by $a\sim b$ $\Leftrightarrow$ $a-b\in I$ is an equivalence relation.

\textbf{Factor Ring}: Let $I$ be ideal of $R$. $R/I:=\{a+I:a\in R\}$ is the set of cosets of $I$. \ \ {\small (i.e. $R/I$ is the set of equivalence classes of $R$ under $\sim$) }

\begin{enumerate}[itemsep=-2pt, topsep=-2pt]
    \item By \textit{well-defined} operators: $(x+I)\dotplus(y+I)=(x+y) + I$ and $(x+I)\cdot(y+I)=xy+I$ \quad $\Rightarrow$ \quad $R/I$ is a ring.
    \item $x+I=y+I \quad \Leftrightarrow \quad x\sim y \quad \Leftrightarrow \quad x-y\in I$ \qquad || \qquad {\footnotesize $R$ is commutative $\Rightarrow$ $R/I$ also.} \qquad || \qquad {\footnotesize $R/I\ne\{0+I\}$ iff $I\ne R$}
    \item The Identity of $R/I$: $1_R+I$ \quad The Zero of $R/I$: $0_R+I$
\end{enumerate}

\textbf{Universal Property of Factor Ring}: Let $R$ be a ring and $I$ be an ideal of $R$. \qquad {\scriptsize ps:$\overline{f}(x+I)=f(x)$}

\begin{enumerate}[itemsep=-2pt, topsep=-2pt]
    \item \textbf{can}: Mapping $can:R\to R/I$ by $x\mapsto x+I$ is $^1$ surjection, $^2$ $ker(can)=I$, $^3$ $can$ is ring homomorphism.
    \item \textbf{$\textbf{f}$}: If $^1f:R\to S$ is \textit{ring homomorphism} and $^2I\subseteq ker(f)$, then \ \ $\exists!$ $^1\overline{f}:R/I\to S$ s.t. $f=\overline{f}\circ can$ is \textit{ring homomorphism}.
    \item \textbf{First Isomorphism Theorem}: If $f:R\to S$ is \textit{ring homomorphism} $\Rightarrow$ $\exists!$ $\overline{f}:R/ker(f)\bij im(f)$ is  (\textit{ring isomorphism}).
\end{enumerate}

\textbf{Universal Property of Quotient Group}: Let $G$ be a group and $H$ be a normal subgroup of $G$. \qquad {\scriptsize ps:$\overline{f}(g+N)=f(g)$}
\begin{enumerate}[itemsep=-2pt, topsep=-2pt]
    \item \textbf{can}: Mapping $can:G\to G/H$ by $x\mapsto xH$ is $^1$ surjection, $^2$ $ker(can)=H$, $^3$ $can$ is group homomorphism.
    \item \textbf{$\textbf{f}$}: If $^1f:G\to S$ is \textit{group homomorphism} and $^2H\subseteq ker(f)$, then \ \ $\exists!$ $^1\overline{f}:G/H\to S$ s.t. $f=\overline{f}\circ can$ is \textit{group homomorphism}.
    \item \textbf{First Isomorphism Theorem}: If $f:G\to S$ is \textit{group homomorphism} $\Rightarrow$ $\exists!$ $\overline{f}:G/ker(f)\bij im(f)$ is  (\textit{group isomorphism}).
\end{enumerate}


\subsection{Modules | Submodules | All of That} % Modules | Submodules | All of That

\textbf{Restrict with Scalar}: Let $f:R\to S$ is a \textit{ring homomorphism}, $f(1_R)=1_S$ and $M$ is a $S-$Module, then $M$ is also a $R-$Module by:

\quad Define the restrict our scalar: $rm:=f(r)m$ \quad $\forall r\in R,m\in M$ \quad {\footnotesize ps: $f(1_R)=1_S$}

\textbf{Free Module}: Let $M$ be a $R-$Module. $M$ is \textit{free} if: $\forall m\in M$, $\exists!$ $r_1,...,r_n\in R$ s.t. $m=r_1m_1+\cdots+r_nm_n$ \quad {\footnotesize ps: $m_1,...,m_n$ is basis of $M$}

\textbf{Coset of Submodule}: Let $N$ submodule of $M$. Then $m+N$ coset of $N$. \quad {\small $\sim$ is defined by $m\sim n$ $\Leftrightarrow$ $m-n\in N$ is an equivalence relation.}

\textbf{Factor Module}: Let $N$ submodule of $M$. $M/N:=\{m+N:m\in M\}$ is the set of cosets of $N$.

\quad {\small ps: All properties of $M/N$ are similar to $R/I$}

\textbf{Universal Property of Module Quotient}: Let $M$ be a module and $N$ be a submodule of $M$. \qquad {\scriptsize ps:$\overline{f}(x+N)=f(x)$}
\begin{enumerate}[itemsep=-2pt, topsep=-2pt]
    \item \textbf{can}: Mapping $can:M\to M/N$ by $x\mapsto x+N$ is $^1$ surjection, $^2$ $ker(can)=N$, $^3$ $can$ is module homomorphism.
    \item \textbf{$\textbf{f}$}: If $^1f:M\to S$ is \textit{module homomorphism} and $^2N\subseteq ker(f)$, then \ \ $\exists!$ $^1\overline{f}:M/N\to S$ s.t. $f=\overline{f}\circ can$ is \textit{module homomorphism}.
    \item \textbf{First Isomorphism Theorem}: If $f:M\to S$ is \textit{module homomorphism} $\Rightarrow$ $\exists!$ $\overline{f}:M/ker(f)\bij im(f)$ is  (\textit{module isomorphism}).
\end{enumerate}

\vspace{2pt}
\np\textbf{Second Isomorphism Theorem for Modules}: Let $N,K$ be submodules of $R$-module $M$ \ \ $\Rightarrow$ \ \ $N/(N\cap K)\cong (N+K)/K$

\quad ps: {\footnotesize consider $f:N\to (N+K)/K$ and then we can find $ker(f)=N\cap K$}

\vspace{2pt}
\np\textbf{Third Isomorphism Theorem for Modules}: Let $N,K$ be submodules of $R$-module $M$ ; $K\subseteq N$ \ \ $\Rightarrow$ \ \ $\frac{M/K}{N/K}\cong M/N$

\quad ps: {\footnotesize consider $f:M/K\to M/N$ and then we can find $ker(f)=N/K$}


\section{Permutation | Determinants | Eigenvalues and Eigenvectors} % Permutation | Determinants | Eigenvalues and Eigenvectors

\subsection{Permutation | Determinants} % Permutation | Determinants

\textbf{Permutation}: A bijection $\sigma:\{1,...,n\}\bij\{1,...,n\}$ is a permutation. \qquad \qquad All permutations of $n$ elements form a group $\mathfrak{S}_n$.

\begin{enumerate}[itemsep=-2pt, topsep=-2pt]
    \item {\scriptsize \textbf{Transposition}: A transposition is a permutation that exchanges two elements. \qquad \textbf{Inversion}: A pair of elements $(i,j)$ is an inversion of $\sigma\in\mathfrak{S}_n$ if $i<j$ but $\sigma(i)>\sigma(j)$}
    \item {\scriptsize \textbf{Length}: The length of a permutation $\sigma$ is the number of inversions. (i.e. $\ell(\sigma):=\left|\ \{(i,j):i<j,\sigma(i)>\sigma(j)\}\ \right|$) \qquad \textbf{Sign}: $\text{sgn}(\sigma):=(-1)^{\ell(\sigma)}$ \qquad $\text{sgn}=1,even \ ; \ \text{sgn}=-1,odd$}
    \item $\text{sgn}(a_1a_2)=-1$ \quad $\text{sgn}(a_1...a_n)=(-1)^{n-1}$ \quad $\text{sgn}(\sigma\tau)=\text{sgn}(\sigma)\text{sgn}(\tau)$ \quad $\big|$ \quad \textbf{Alternating Group}: $A_n:=\{\sigma\in\mathfrak{S}_n:\text{sgn}(\sigma)=1\}$
    \item \textbf{Graph Meaning of Inversion}: {\small Inversion is \# edges that cross each other in the graph of permutation. {\footnotesize (i.e. 画出的图中,线段交叉的次数)}}
\end{enumerate}

\textbf{Determinant}: For matrix $A_{n\times n}$, with $A_{ij}=a_{ij}$. \quad $\det(A):=\sum_{\sigma\in\mathfrak{S}_n}\text{sgn}(\sigma)a_{1\sigma(1)}\cdots a_{n\sigma(n)}$ (\textbf{Leibniz Formula}) \qquad $\det(I_0):=1$

\textbf{Geometric Meaning of Determinant}: Let $\text{area}(U)$ denote the area|volume of $U$. \quad Let $A$ denote a matrix.
\begin{enumerate}[itemsep=-2pt, topsep=-2pt]
    \item {\footnotesize $\det(A)$ 对 U 操作后的面积|体积 = $|\det(A)|\times\text{area}(U)$} \quad 2. \ {\footnotesize $\text{sgn}(\det A)$ 决定了方向是否改变(+1不变,-1变). {\scriptsize (i.e. 顺逆时针变化,左右|上下变化,手性变化)}}
\end{enumerate}

\vspace{3pt}
\textbf{Bilinear|Multilinear form}: {\small $U,V,V_i,W$ be F-vector space. \quad A mapping \ \ $H: U\times V\to W$ \ \ or \ \ $H:V_1\times \cdots\times V_n\to W$ \ \ is \ \ \textit{bilinear | multilinear} if:}

\vspace{2pt}
\noindent
\begin{minipage}{0.4\linewidth}
    \begin{enumerate}[itemsep=-2pt, topsep=-2pt]
        \item $H(\lambda u,v)=\lambda H(u,v)$
        \item $H(u+v,w)=H(u,w)+H(v,w)$
        \item $H(u,\lambda v)=\lambda H(u,v)$
        \item $H(u,v+w)=H(u,v)+H(u,w)$
    \end{enumerate}
\end{minipage}%
\hfill
\begin{minipage}{0.6\linewidth}
    \begin{enumerate}[itemsep=-2pt, topsep=-2pt]
        \item $H(u_1,...,\lambda v_i,...,u_n)=\lambda H(u_1,...,v_i,...,u_n)$ \quad $\forall i$
        \item $H(u_1,...,v_i+v_j,...,u_n)=H(u_1,...,v_i,...,u_n)+H(u_1,...,v_j,...,u_n)$ \quad $\forall i$ \\
        {\scriptsize (左边bilinear,右边multilinear)}
    \end{enumerate}
\end{minipage}

\vspace{3pt}
\quad H is \textbf{Symmetric} if (bilinear): $^1U=V$, \quad $^2H(u,v)=H(v,u)$ \ \ $\forall u,v\in U$

\hspace{78pt} if (multilinear): $^1V_i$ same, \quad $^2H(v_1,...,v_n)=H(v_{\sigma(1)},...,v_{\sigma(n)})$ \ \ $\forall \sigma\in\mathfrak{S}_n$

\quad H is \textbf{Alternating|Antisymmetric} if (bilinear): $^1U=V$, \quad $^2H(u,u)=0$ \ \ $\forall u\in U$

\hspace{153pt} if (multilinear): $^1V_i$ same, \quad $^2H(v_1,...,v_n)=0$ \ \ $\forall v_i=v_j$ \quad {\scriptsize (i.e. 只要存在两个及以上相同的, $H$结果为0)}

\quad \textbf{Lemma}: If $H$ is \textit{alternating}, then \quad $H(u,v)=-H(v,u)$ \qquad $H(v_1,...,v_i,...,v_j,...,v_n)=-H(v_1,...,v_j,...,v_i,...,v_n)$ \qquad {\scriptsize ($\Leftarrow$不一定成立)}

\textbf{Property of Determinant}: Let $A,B$ be $n\times n$ matrices. \quad $F$ be field. \quad $R$ be \textit{commutative ring}.
\begin{enumerate}[itemsep=-2pt, topsep=-2pt]
    \item \textbf{Unique on Field}: $\det:F^n\times \cdots\times F^n\to F$ \textit{or} $\det: Mat(n;F) \to F$ is the $^1$\textit{unique} $^2$\textit{alternating} $^3$\textit{multilinear form} s.t. $\det(I_n)=1_F$
    \item \textbf{Invertible on Field}: For $Mat(n;F)$, $A$ is invertible $\Leftrightarrow$ $\det(A)\ne0$ \qquad {\scriptsize 对于communitative ring, 这个结论成立如果$\det(A)$在$R$中有逆}
    \item \textbf{Operations}: {\scriptsize If $R$ is \textit{commutative ring}}, then $\det(AB)=\det(A)\det(B)$ \quad $\det(A^T)=\det(A)$ \quad $\det(A^{-1})=\det(A)^{-1}$
    \item \textbf{Similar}: {\scriptsize If $R$ is \textit{commutative ring}}, $A\sim B$ $\Rightarrow$ $\det(A)=det(P^{-1}BP)=\det(B)$ \qquad Thus, we can define: $det(f)$ for $f:V\to V$
\end{enumerate}



\section{Inner Product Spaces | Orthogonal Complement / Proj | Adjoints and Self-Adjoint} % Inner Product Spaces | Orthogonal Complement / Proj | Adjoints and Self-Adjoint


\section{Jordan Normal Form | Spectral Theorem} % Jordan Normal Form | Spectral Theorem


\end{document}
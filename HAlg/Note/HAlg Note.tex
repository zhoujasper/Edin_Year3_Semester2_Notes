\documentclass[9pt]{article}

\usepackage{graphicx} % Required for inserting images
\usepackage{amsmath}
\usepackage{amsfonts}
\usepackage{ctex}
\usepackage{enumitem}
\usepackage{longtable}
\usepackage{makecell} % 换行

% 使用分栏宏包
\usepackage{multicol} 
\setlength{\columnseprule}{0.4pt} % 分割线

% 设置字体
\usepackage{unicode-math}
\setmainfont{Cambria}
\setmathfont{Cambria Math}

% 调整页面布局
\usepackage[a4paper, top=0.7cm, bottom=1cm, left=0.7cm, right=.7cm]{geometry}
\setlength{\footskip}{15pt}

% 设置页脚/页眉
\usepackage{fancyhdr}
\fancyfoot[C]{Copyright By Jingren Zhou | Page \thepage}
\fancyhead[]{}
\pagestyle{fancy}
% 去除线
\renewcommand{\headrulewidth}{0pt}
\renewcommand{\footrulewidth}{0pt}

% 设置 section/subsection 之间的行间距
\usepackage{titlesec}
\titlespacing*{\section}{-2pt}{-2pt}{-2pt}
\titlespacing*{\subsection}{-2pt}{-2pt}{0pt}

% 调整标题上下间距
\usepackage{titling}
\setlength{\droptitle}{-2.4cm} % 负值表示向上移动

\newcommand{\oto}{\hookrightarrow}
\newcommand{\onto}{\twoheadrightarrow}
\newcommand{\bij}{\stackrel{\sim}{\rightarrow}}

% 设置标题,作者,时间
\title{HAlg Note}
\author{}
\date{}

% 正文
\begin{document}


% 标题
\maketitle
\thispagestyle{fancy}
\vspace{-3.5cm}

% 字体大小
\fontsize{10pt}{11pt}\selectfont
\setlength{\parindent}{8pt}


\section{Basic Knowledge} % Basic Knowledge

\textbf{Def of Matrix}: A mapping from $\{1,...,n\}\times\{1,...,m\}$ to a field $F$ is called a $n\times m$ matrix over $F$.

$\cdot$ The set of all $n\times m$ matrices over $F$ is denoted by $Mat(n\times m;F):=Maps(\{1,...,n\}\times\{1,...,m\},F)$. \quad \textbf{Square Matrix}: $Mat(n;F)$

\textbf{Solution Sets of Inhomogeneous Systems of Linear Equations}: {\small Solution = 特解(Particular Solution) + 通解(Homogeneous solution)}

\textbf{Def of Group $(G,*)$}: A set $G$ with a operator $*$ is a group if: {\small \textbf{Closure}: $\forall g,h\in G$, $g*h\in G$; \textbf{Associativity}: $\forall g,h,k\in G$, $(g*h)*k=g*(h*k)$;}

\hspace{85pt} {\small \textbf{Identity}: $\exists e\in G$, $\forall g\in G$, $e*g=g*e=g$; \textbf{Inverse}: $\forall g\in G$, $\exists g^{-1}\in G$, $g*g^{-1}=g^{-1}*g=e$.}

$\cdot$ \textbf{Properties of Group}: If $G,H$ are groups, then $G\times H$ also.

\textbf{Field $(F)$}: A set $F$ is a field with two operators: \quad {\scriptsize(addition)}$+:F\times F\to F;(\lambda,\mu)\to\lambda+\mu$ \quad {\scriptsize(multiplication)}$\cdot:F\times F\to F;(\lambda,\mu)\to\lambda\mu$ \ \ if:

\quad \quad \quad \quad $(F,+)$ and $(F\setminus\{0_F\}, \ \cdot \ )$ are abelian groups with identity $0_F,1_F$. \quad and \quad $\lambda(\mu+\nu)=\lambda\mu+\lambda\nu$ \quad \quad $e.g. Fields:\mathbb{R},\mathbb{C},\mathbb{Q},\mathbb{Z}/p\mathbb{Z}$

\textbf{Notation of 1-1,onto,bij}: For function $f:V\to W$. \quad \textbf{1-1}: $V\oto W$ \quad \textbf{onto}: $V\onto  W$ \quad \textbf{bijection}: $V\bij W$ {\tiny (ps: bij证法:1.def;2.$f f^{-1}=I,f^{-1}f=I$)}

\textbf{Projections ($pr_i$)}: {\small $pr_i:X_1\times X_2\times\cdots\times X_n \to X_i$:$(x_1,...,x_n)\mapsto x_i$} \quad \textbf{Canonical Injections}: {\small $in_i:X_i\to X_1\times X_2\times\cdots\times X_n$: $x\mapsto (0,...,x,0,...,0)$}


\section{Vector Spaces} % Vector Spaces

\subsection{Vector Spaces | Product of Sets | Vector Subspaces | Power, Union, Intersection of Sets}

\textbf{$F$-Vector Space (V)}: A set $V$ over a field $F$ is a vector space if: \quad $V$ is an abelian group $V=(V,\dotplus)$ and $\forall \ \vec{v},\vec{w}\in V$ \ \ $\lambda,\mu\in F$

\quad \quad \quad \quad a map $F\times V\to V:(\lambda,\vec{v})\to\lambda\vec{v}$ satisfies: \textbf{I}: $\lambda(\vec{v}\dotplus\vec{w})=(\lambda\vec{v})\dotplus(\lambda\vec{w})$ \quad \textbf{II}: $(\lambda+\mu)\vec{v}=(\lambda\vec{v})\dotplus(\mu\vec{v})$

\quad \quad \quad \quad \textbf{III}: $\lambda(\mu\vec{v})=(\lambda\mu)\vec{v}$ \quad \textbf{IV}: $1_F\vec{v}=\vec{v}$ \hspace{30pt} {\scriptsize ps:If $\lambda\vec{v}=\vec{0}$, then $\lambda=0$ or $\vec{v}=\vec{0}$ or both. \hspace{30pt} \textbf{Trivial Vector Space}: $V=\vec{0}$}

$\cdot$ If $V,W$ are $F$-vector spaces, then $V\oplus W$ is also. \ \ \ ps: $V\oplus W:=V\times W$

\textbf{Vector Subspace (U)}: $U\subseteq V$ is a subspace of $V$ if: \quad \textbf{I}. $\vec{0}\in U$ \quad \textbf{II}. $\forall \vec{u},\vec{v}\in U$, $\forall \lambda\in F$: $\vec{u}+\vec{v}\in U$ and $\lambda\vec{u}\in U$ \quad (or: $\lambda\vec{u}+\mu\vec{v}\in U$)

\begin{enumerate}[itemsep=-2pt, topsep=-2pt]
    \item If $U_1,U_2$ are subspaces of $V$. Then $U_1\cap U_2$ and $U_1+U_2$ are also. \quad {\scriptsize ps: $U_1+U_2:=\langle U_1\cup U_2\rangle$}
    \item \textbf{Vector Subspace Generated by T ($\langle T \rangle$)}: {\small If $T$ is a subset of a $F$-vector space $V$. \ $\Rightarrow$ \ $\langle T \rangle$ is the smallest subspace of $V$ containing $T$.} \\
    \quad Also, we can get: $\langle T \rangle=span(T):=\{\sum_ic_i\vec{v_i}: \vec{v_i}\in T,c_i\in F\}$ \quad \quad \quad $\forall \vec{v}\in\langle T \rangle,\langle T \cup\{\vec{v}\}\rangle=\langle T \rangle$
    \item \textbf{Generating/Spanning Set}: If $\langle T \rangle=V$. \ $\Rightarrow$ \ $T$ is a generating set of $V$. \quad \textbf{Finitely Generated}: $\exists$ T finite set, s.t. $V=\langle T\rangle$
\end{enumerate}

\textbf{Free Vector Space on the Set X}: Set $X$, 将$X$中每一个元素都视为 基, then $\{\sum_{x\in X}a_xx:a_x\in F,F \ is \ field\}$ is FVS on $X$.

\textbf{Functional Vector Space}: If $X$ be a set and $F$ be field. Then $Maps(X,F)$ is a $F$-Vector Space. \quad \quad {\tiny ps: 'almost all': all but finitely many (全部,但可以有有限个除外)}

$\cdot$ $F\langle X\rangle:=\{f:X\to F \ | \ f(x)=0 \ for \ almost \ all \ x\in X\}$ \quad \quad \quad {\footnotesize ps: $F\langle X\rangle$ is a subspace of $Maps(X,F)$} \ \ \ ?没写完!

\textbf{Power of Set $\mathcal{P}(X)$}: If X is a set, then $\mathcal{P}(X):=\{U:U\subseteq X\}$ {\scriptsize (set of all subsets)} \quad \quad {\small ps: $\mathcal{U}\subseteq\mathcal{P}(X)$ \ $\Rightarrow$ \ $U$ is called a \textbf{system of subsets of $X$}}.

\begin{enumerate}[itemsep=-2pt, topsep=-2pt]
    \item \textbf{Empty System of subsets of X}: Empty System of subsets of X $:=\emptyset\in\mathcal{P}(X)$ \ {\scriptsize (NOT $\{\emptyset\}$)} \quad \quad \quad \star \ $\bigcap\emptyset=X$ \quad and \quad $\bigcup\emptyset=\emptyset$ \ \star
    \item \textbf{Union}: {\small For $\mathcal{U}\subseteq\mathcal{P}(X)$, $\bigcup_{U\in\mathcal{U}}U:=\{x\in X:\exists U\in\mathcal{U} \ s.t. \ x\in U\}$} \quad \textbf{Intersection}: {\small For $\mathcal{U}\subseteq\mathcal{P}(X)$, $\bigcap_{U\in\mathcal{U}}U:=\{x\in X:\forall U\in\mathcal{U},x\in U\}$}
\end{enumerate}


\subsection{Linear Independence | Basis | Dimension}

\textbf{Linearly Independent}: $L=\{\vec{v_1},\vec{v_2},...,\vec{v_r}\}$ is linearly independent if: \ $\forall c_1,...,c_r\in F$, $c_1\vec{v_1}+...+c_r\vec{v_r}=\vec{0}$ \ $\Rightarrow$ \ $c_1=...=c_r=0$.

\textbf{Linearly Dependent}: $L$ is linearly dependent if: $\exists \alpha_1,...,\alpha_r$ not all zero \ s.t. \ $\alpha_1\vec{v_1}+...+\alpha_r\vec{v_r}=\vec{0}$

\textbf{Basis}: A basis of a vector space $V$ is a linearly \textit{independent} \textit{generating set} of $V$. \quad \quad (Finitely generated $\Leftrightarrow$ $\exists$ finite basis.)

\begin{enumerate}[itemsep=-2pt, topsep=-2pt]
    \item subset $E$ is a basis \ $\Leftrightarrow$ \ $E$ is minimal generating sets \ $\Leftrightarrow$ \ $E$ is maximal linearly independent sets.
    \item \textbf{Fundamental Estimate of Linear Algebra}: Linearly independent sets $\subseteq$ basis $\subseteq$ generating sets.
    \item $(\vec{v}_i)_{i\in I}$ is a basis of $V$ \ $\Leftrightarrow$ \ $\forall \vec{v}\in V$, $\exists !$ $c_i\in F$ (almost all of $c_i$ are zero) \ s.t. \ $\vec{v}=\sum_{i\in I}c_i\vec{v}_i$
\end{enumerate}

\textbf{Family of Elements of $A$ Indexed by $I$}: $(a_i)_{i\in I}:=func \ f:I\to A$ with $i\mapsto a_i$. \quad {\tiny e.g. $f(0)=1,f(1)=2,f(2)=3$ 可以用 $(a_i)_{i\in\{0,1,2\}},a_0=1,a_1=2,a_2=3$ 代替}

$\cdot$ If $\{\vec{v}_i:i\in I\}$ is generating set of $V$, then $(\vec{v_i})_{i\in I}$ is called a generating set. \quad \quad {\footnotesize (同理对: $(\vec{v}_i)_{i\in I}$ is \textbf{basis indexed by $i\in I$})}

\textbf{Linear Combinations of Basis}: Let $F$ be a field, \ family $(\vec{v}_i)_{1\leq i\leq r}$, \ $V$ is vector space. \ \ $\Phi:F^r\to V$ with $(c_1,...,c_r)\mapsto c_1\vec{v}_1+...+c_r\vec{v}_r$:

\begin{enumerate}[itemsep=-2pt, topsep=-2pt]
    \item \textbf{I}.$(\vec{v}_i)_{1\leq i\leq r}$ is generating set $\Leftrightarrow$ $\Phi$ is onto. ($F^r\onto V$) \quad \quad \quad \textbf{II}.$(\vec{v}_i)_{1\leq i\leq r}$ is linearly independent $\Leftrightarrow$ $\Phi$ is 1-1. ($F^r\oto V$)
    \item $(\vec{v}_i)_{1\leq i\leq r}$ is basis $\Leftrightarrow$ $\Phi$ is bijection. ($F^r\bij V$)
\end{enumerate}

\textbf{Steinitz Exchange Theorem}: Let $V$ be vector space. $L$ is linearly independent set, $E$ is generating set. \ $\Rightarrow$ \ $\exists \ \text{1-1} \ \phi:L\oto E$ s.t.

\hspace{130pt} $(E\setminus\phi(L))\cup L$ is generating set. {\tiny i.e. $E$中的一部分元素可以完全由$L$中的元素线性表示出来.(即$L$中的元素可以替换$E$中的部分元素)}

\textbf{Dimension}: Dimension of $F$-vector space is $\dim_FV:=$ \# basis (i.e. cardinality of basis). \quad \quad \quad {\small e.g. $\dim_FF^n=n$}

$\cdot$ Let $V$: Vector Space. $L$ LI set, $E$ generating set. \ \textbf{I}. $\dim L\leq \dim V\leq\dim E$ \ \textbf{II}. If $|L|=\dim V$ ($|E|=\dim V$), then $L$ ($E$) is basis.

$\cdot$ \textbf{Dimension Theorem}: Let $V$: Vector Space. $U,W$: Subspaces. \ \textbf{I}. $\dim(U+W)=\dim U+\dim W-\dim(U\cap W)$ \ \textbf{II}. $\dim U\leq\dim V$


\subsection{Linear Maps | Rank-Nullity Theorem}

\textbf{Linear Maps}: $f:V\to W$ {\small (V,W vector spaces) is \textbf{$F$-linear} or \textbf{homomorphism} if: \quad $f(\vec{v}+\vec{w})=f(\vec{v})+f(\vec{w})$ \ $f(\lambda\vec{v})=\lambda f(\vec{v})$ \quad {\footnotesize $\forall \vec{v},\vec{w}\in V;\forall\lambda\in F$}}

\textbf{Isomorphism}: {\small Linear map $f:V\to W$ is bij. \ \textbf{Endomorphism (End)}: Linear map $f:V\to V$. \ \textbf{Automorphism (Aut)}: Isomorphism. $f:V\to V$.}

\textbf{General Linear Group / Automorphism Group}: $GL(V)=Aut(V):=\{f:V\to V \ | \ f \ is \ isomorphism\}$ {\scriptsize (subspace)}

\textbf{Fixed Point}: For $f:X\to X$, if $f(x)=x$, then $x$ is fixed point of $f$. \quad \textbf{Set of Fixed Points}: $X^f:=\{x\in X:f(x)=x\}$

\textbf{Notation $\oplus$}: Let $U,W$ be subspaces of $V$. \quad $V_1,...,V_n$ be subspaces of $V$. \quad \quad $V_1+\cdots+V_n:=\langle V_1\cup\cdots\cup V_n\rangle$:

\begin{enumerate}[itemsep=-2pt, topsep=-2pt]
    \item \textbf{Basic Case}: If $f:U\times W\bij V$ by $(\vec{u},\vec{v})\mapsto\vec{u}+\vec{v}$ then we write: $V=U\oplus W$ and called $U,W$ \textbf{Complementary}.
    \item \textbf{Special Case}: If $f:V_1+\cdots+ V_n\to V$ by $x\mapsto x$ ? 不对把? P18 没写完
\end{enumerate}

\textbf{Classification of Vector Spaces}: For Vector Space $V$, \ $\dim V = \dim F^n=n$ $\Leftrightarrow$ $\exists\phi:F^n\bij V$ is isomorphism.

\end{document}
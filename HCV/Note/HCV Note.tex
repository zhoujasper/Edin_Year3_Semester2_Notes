\documentclass[9pt]{article}

\usepackage{graphicx} % Required for inserting images
\usepackage{amsmath}
\usepackage{amsfonts}
\usepackage{ctex}
\usepackage{enumitem}
\usepackage{longtable}
\usepackage{makecell} % 换行

% 使用分栏宏包
\usepackage{multicol} 
\setlength{\columnseprule}{0.4pt} % 分割线

% 设置字体
\usepackage{unicode-math}
\setmainfont{Cambria}
\setmathfont{Cambria Math}

% 调整页面布局
\usepackage[a4paper, top=0.7cm, bottom=1cm, left=0.7cm, right=.7cm]{geometry}
\setlength{\footskip}{15pt}

% 设置页脚/页眉
\usepackage{fancyhdr}
\fancyfoot[C]{Copyright By Jingren Zhou | Page \thepage}
\fancyhead[]{}
\pagestyle{fancy}
% 去除线
\renewcommand{\headrulewidth}{0pt}
\renewcommand{\footrulewidth}{0pt}

% 设置 section/subsection 之间的行间距
\usepackage{titlesec}
\titlespacing*{\section}{0pt}{-2pt}{-2pt}
\titlespacing*{\subsection}{0pt}{-2pt}{-2pt}

% 调整标题上下间距
\usepackage{titling}
\setlength{\droptitle}{-2.4cm} % 负值表示向上移动

% 设置标题,作者,时间
\title{HCV Note}
\author{}
\date{}

% 正文
\begin{document}

\newcommand{\np}{{\tiny $^{^{^\ominus}}$}}

% 标题
\maketitle
\thispagestyle{fancy}
\vspace{-3.5cm}

% 字体大小
\fontsize{10pt}{11pt}\selectfont
\setlength{\parindent}{8pt}

\section{Basic Knowledge} % Basic Knowledge

\textbf{Useful Complex Number Properties}: $|Re(z)|,|Im(z)|\leq |z|$ \quad $Re(z)=\frac{z+\overline{z}}{2},Im(z)=\frac{z-\overline{z}}{2i},|z|^2=z\overline{z}$ \quad In \textit{circle}, $\overline{z}=|z|^2z^{-1}$

\textbf{Triangle (Reverse) Inequality}: $|z_1+z_2|\leq |z_1|+|z_2|$ \quad $||z_1|-|z_2||\leq|z_1-z_2|$ \quad \quad \quad {\scriptsize\np ($Re(zw)=0\Leftrightarrow \overline{zw}=-zw$;$Im(zw)=0\Leftrightarrow zw=\overline{zw}$)}

\textbf{Argument}: $\arg(z):=\{\theta:z=|z|e^{i\theta}\}=\{Arg(z)+2\pi k:k\in\mathbb{Z}\}$ \quad \textbf{Principle Value of Argument}: $Arg(z)\in(-\pi,\pi]$

$\cdot$ \textbf{Operations on Argument}: $arg(z_1z_2)=arg(z_1)+arg(z_2)$ \quad $arg\left(\frac{z_1}{z_2}\right)=arg(z_1)-arg(z_2)$ \quad $arg(\overline{z})=-arg(z)$


\section{Holomorphic Functions} % Holomorphic Functions

\subsection{Open/Closed Set | Limit Point | limit of Sequence | Continuous of Function} % Open/Closed Set | Limit Point | limit of Sequence | continuous of Function

\textbf{Open/Closed/Punctured $\varepsilon$-disc}: {\small $D_\varepsilon(z_0):=\{z\in\mathbb{C}:|z-z_0|<\varepsilon\}$ \quad $\overline{D}_\varepsilon(z_0):=\{z\in\mathbb{C}:|z-z_0|\leq\varepsilon\}$ \quad $D_\varepsilon'(z_0):=\{z\in\mathbb{C}:0<|z-z_0|<\varepsilon\}$}

\textbf{Open/Closed Set in $\mathbb{C}$}: {\small $U\subset\mathbb{C}$ is \textbf{open} if $\forall z_0\in U,\exists\varepsilon>0,D_\varepsilon(z_0)\subseteq U$ \quad $U$ is \textbf{closed} if $\mathbb{C}\setminus U$ is open} \quad \textbf{Lemma}: $D_\varepsilon,D'_\varepsilon$ open, $\overline{D}_\varepsilon$ closed.

\textbf{Limit Point of S}: $z_0\in\mathbb{C}$ is a limit point of $S$ if: \ $\forall\varepsilon>0,D'_\varepsilon(z_0)\cap S\neq\emptyset${\tiny 聚点} \quad \quad \textbf{Bounded}: $S$ is bounded if $\exists M>0$ s.t. $|z|\leq M,\forall z\in S$

\textbf{Closed of Set S}: $\overline{S}:=$所有S的limit point和S的点. \quad \textbf{Property}: Let $S\subseteq\mathbb{C}$, then $S$ is closed \ $\Leftrightarrow$ \ $S=\overline{S}$.

\textbf{Limit of sequence}: Sequence $(z_n)_{n\in\mathbb{N}}$ has limit $z$ if $\forall\varepsilon>0,\exists N\in\mathbb{N}$ s.t. $\forall n\geq N\Rightarrow |z_n-z|<\varepsilon$. {\scriptsize limit rules 依旧成立}

\begin{enumerate}[itemsep=-2pt, topsep=-2pt]
    \item \textbf{Lemma|Important}: $\lim z_n=z$ \ $\Leftrightarrow$ \ $\lim Re(z_n)=Re(z)$ and $\lim Im(z_n)=Im(z)$
    \item \textbf{Cauchy}: Sequence $(z_n)_{n\in\mathbb{N}}$ is cauchy if: \ $\forall\varepsilon>0,\exists N\in\mathbb{N}$ s.t. $\forall m,n\geq N\Rightarrow |z_m-z_n|<\varepsilon$ \quad \textbf{Lemma}: Cauchy $\Leftrightarrow$ convergent.
    \item \textbf{Lemma|Closed of Set}: $S\subseteq\mathbb{C},z\in\mathbb{C}$. $\Rightarrow$ [ \ $z\in\overline{S}$ \ $\Leftrightarrow$ \ $\exists$ sequence $(z_n)_{n\in\mathbb{N}}\in S$ s.t. $\lim z_n=z$ \ ]
    \item \textbf{Bolzano-Weierstrass}: Every bounded sequence in $\mathbb{C}$ has a convergent subsequence.
\end{enumerate}

\textbf{Complex Functions}: $\forall f:\mathbb{C}\to\mathbb{C}$ we can write it as: $f(z)=f(x+iy)=u(x,y)+iv(x,y)$ where $u,v:\mathbb{R}^2\to\mathbb{R}$

\textbf{Limit of Function}: $a_0\in\mathbb{C}$ is the limit of $f$ at $z_0$ if: \ $\forall\varepsilon>0,\exists\delta>0$ s.t. $0<|z-z_0|<\delta\Rightarrow |f(z)-a_0|<\varepsilon$ \quad {\scriptsize limit rules 依旧成立}

$\cdot$ \textbf{Lemma|Important}: $\lim_{z\to z_0} f(z)$ $\Leftrightarrow$ $\lim_{(x,y)\to(x_0,y_0)} u(x,y)=Re(a_0)$ and $\lim_{(x,y)\to(x_0,y_0)} v(x,y)=Im(a_0)$

$\cdot$ \textbf{Useful Formula}: $\lim_{z\to z_0}g(\ \overline{z} \ )=\lim_{z\to\overline{z_0}}g(z)$

\textbf{continuous of Function}: $f$ is continuous at $z_0$ if: \ $\forall\varepsilon>0,\exists\delta>0$ s.t. $|z-z_0|<\delta\Rightarrow |f(z)-f(z_0)|<\varepsilon$ \quad {\scriptsize continuous rules 依旧成立}

\begin{enumerate}[itemsep=-2pt, topsep=-2pt]
    \item \textbf{Lemma|Important}: $f$ is continuous at $z_0$ $\Leftrightarrow$ $u,v$ are continuous at $(x_0,y_0)$
    \item \textbf{`Extreme Value Theorem'}: $f$ is continuous on a closed and bounded set $S\subseteq\mathbb{C}$, then $f(S)$ is closed and bounded.
    \item \textbf{Lemma|continuous$\Leftrightarrow$open}: $f$ is continuous $\Leftrightarrow$ $\forall$ open set $U$, preimage $f^{-1}(U):=\{z\in\mathbb{C}|f(z)\in U\}$ is open.
\end{enumerate}


\subsection{Differentiable | Holomorphic Function | C-R Equation} % Differentiable | Holomorphic Function | C-R Equation

\textbf{Differentiable}: Let $z_0\in\mathbb{C}$ and $U\subseteq\mathbb{C}$ be neighborhood of $z_0$, then $f:U\to\mathbb{C}$ is differentiable at $z_0$ if: \ $\lim_{z\to z_0}\frac{f(z)-f(z_0)}{z-z_0}$ exists.

$\cdot$ \textbf{I}. $f$ is differentiable $\Rightarrow$ $f$ is continuous. \quad \quad \textbf{II}. \textbf{Holomorphic} $\Leftrightarrow$ {\small Differentiable + neighborhood} {\tiny (除非是一个点时不成立,$|z|$)} \quad {\scriptsize diff rules + chain rule 成立}

\textbf{Cauchy-Riemann Equations}: If $z_0=x_0+iy_0$, $f(z)=u(x,y)+iv(x,y)$ is differentiable at $z_0$ \ \ $\Rightarrow$ \ \ $u_x=v_y,v_x=-u_y$ \textbf{at} $(x_0,y_0)$.

$\cdot$ If $z_0=x_0+iy_0$, $f=u+iv$ satisfies: $^1u,v$ are \textit{continuously differentiable} on a neighborhood of $(x_0,y_0)$ and:

\hspace{162pt} $^2u,v$ satisfies Cauchy-Riemann Equations at $(x_0,y_0)$. \ \ $\Rightarrow$ \ \ $f$ is differentiable at $z_0$.

$\cdot$ {\footnotesize ps: 常见可导复数函数: $\exp(z),\sin z,\cos z,\log z,z^\alpha,\text{polynomial},\sinh,\cosh,\Gamma(z),|z|^2\text{(at 0)}$,\text{constant}} \quad \quad {\footnotesize ps: 常见不可导复数函数: $\overline{z},|z|\cdot\overline{z},Re(z),Im(z),\text{Arg}(z)$}

\textbf{Harmonic Function}: $h:\mathbb{R}^2\to\mathbb{R}$ is harmonic if: $\forall(x,y)\in\mathbb{R}^2$ \ $\frac{\partial^2h}{\partial x^2}+\frac{\partial^2h}{\partial y^2}=0$ {\footnotesize (\textbf{Laplace Equation})}

$\cdot$ \textbf{Lemma}: If $f=u+iv$ is holomorphic on $\mathbb{C}$ {\tiny (and $u,v$ are twice \textit{continuously differentiable}) 可以不用}, \ \ $\Rightarrow$ \ \ $u,v$ are harmonic.

\textbf{Harmonic Conjugate}: {\small Let $u,v:U\to\mathbb{R},U\subseteq\mathbb{R}^2 $ be harmonic functions. $u,v$ are harmonic conjugate if: $f=u+iv$ is holomorphic on $U$.}

\textbf{Properties of Polynomial}: {\small The domain of rational function and polynomial are always open.} \quad \quad \textbf{Lemma}: If $P(z_0)=0$ then $P(\overline{z_0})=0$

\textbf{First-order Operator $\partial$}: $\partial:=\frac{1}{2}\left(\frac{\partial}{\partial x}-i\frac{\partial}{\partial y}\right)$ \quad \quad $\overline{\partial}:=\frac{1}{2}\left(\frac{\partial}{\partial x}+i\frac{\partial}{\partial y}\right)$ \quad || \quad $f=u+iv$ satisfies C-R Equations \ $\Leftrightarrow$ \ $\overline{\partial}f=0$

\textbf{sin/cos Functions}: $\sin z:=\frac{e^{iz}-e^{-iz}}{2i}$ \quad \quad $\cos z:=\frac{e^{iz}+e^{-iz}}{2}$ \quad \quad \textbf{Exponential Function}: $\exp(z)=e^x(\cos(y)+i\sin(y))$

\begin{enumerate}[itemsep=-2pt, topsep=-2pt]
    \item $\sin(x+iy)=\sin x\cosh y+i\cos x\sinh y$ \quad $\cos(x+iy)=\cos x\cosh y-i\sin x\sinh y$
    \item $\sin(z+w)=\sin(z)\cos(w)+\cos(z)\sin(w)$ \quad $\cos(z+w)=\cos(z)\cos(w)-\sin(z)\sin(w)$ 
    \item $\sin^2z+\cos^2z=1$ \quad $\sin(z+\frac{\pi}{2})=\cos(z)$ \quad $\sin(z+2k\pi)=sin(z)$ \quad $\cos(z+2k\pi)=\cos(z)$ \quad \quad \star $\sin z,\cos z$ NOT bounded.
\end{enumerate}

\textbf{Hyperbolic Functions}: $\sinh z:=\frac{\exp(z)-\exp(-z)}{2}$ \quad \quad $\cosh z:=\frac{\exp(z)+\exp(-z)}{2}$ \quad \quad || \quad \quad $\sinh(iz)=i\sin z$ \quad $\cosh(iz)=\cos z$

\textbf{Logarithm}: Define \textit{multivalued function}: $\log{z}:=\{w\in\mathbb{C}:\exp{w}=z\}$ \quad \quad \quad \textbf{Principal Branch}: $Log(z):=\ln|z|+iArg(z)$

\begin{enumerate}[itemsep=-2pt, topsep=-2pt]
    \item \textbf{I}. $\log(z)=\ln |z|+i\arg{z}=\{\ln |z|+iArg(z) +i2\pi k:k\in\mathbb{Z}\}$ \quad \quad \textbf{II}. $\log(zw)=\log(z)+\log(w)$ \quad \quad \textbf{III}. $\log(1/z)=-\log(z)$
    \item \textbf{Branch of Logarithm}: $Log_{\phi}(z):=\ln|z|+iArg_{\phi}(z)$ \quad \quad $Log_\phi(z)$ is holomorphic on $D_{\phi}$ \\
    \noindent
    \makebox[\textwidth][c]{
        \parbox{0.83\textwidth}{
            \raggedleft
            \includegraphics[width=0.28\textwidth]{Stereographic-projection.png}
        }
    }
    \vspace{-3cm}
    \item If $g:U\to\mathbb{C}$, then $Log_\phi(g(z))$ is holomorphic on $g^{-1}(D_{\phi})\cap U$
    \item $Log(z)$ \textit{not continuous} on $\mathbb{C}$. \quad \quad $Log(z)$ not continuous on $Re(z)\leq 0,Im(z)=0$. \\
    \textbf{Remark}: $\log(x)+\log(x)\ne 2\log(x)$
\end{enumerate}

\textbf{Branch Cut|Cut Plane}: \textit{Branch Cut} $L_{z_0,\phi}:=\{z\in\mathbb{C}:z=z_0+re^{i\phi},r\geq0\}$

$\cdot$ \textit{Cut Plane}: $D_{z_0,\phi}:=\mathbb{C}\setminus L_{z_0,\phi}$ \quad {\footnotesize $L_{\phi}=L_{0,\phi};D_{\phi}=D_{0,\phi}$}

$\cdot$ If $Log_{\phi}(z)$ is holomorphic on $D_{\phi}$, then $Log_{\phi}(z-a)$ is holomorphic on $D_{a,\phi}$

\textbf{Branch of Argument}: $Arg_{\phi}(z):=z${\small 的辐角,但是角度限制在:} $\phi<Arg_\phi(z)\leq \phi+2\pi$. \quad \quad ps: {\footnotesize $Arg_{-\pi}(z)=Arg(z)$}

\vspace{-10pt}
\begin{longtable}{cc|cc|cc|cc|cc|cc|cc}
    $f(z)$ & $f'(z)$ & $f(z)$ & $f'(z)$ & $f(z)$ & $f'(z)$ & $f(z)$ & $f'(z)$ & $f(z)$ & $f'(z)$ & $f(z)$ & $f'(z)$ & $f(z)$ & $f'(z)$ \\
    \hline
    $z^n$ & $nz^{n-1}$ & $\exp(z)$ & $\exp(z)$ & $\sin(z)$ & $\cos(z)$ & $\cos(z)$ & $-\sin(z)$ & $\sinh(z)$ & $\cosh(z)$ & $\cosh(z)$ & $\sinh(z)$ & $Log_\phi z$ & $\frac{1}{z}$ {\tiny $z\in D_{\phi}$} \\
\end{longtable}
\vspace{-10pt}

\textbf{Complex Powers}: $z^{\alpha}:=\{\exp(\alpha w):w\in\log(z)\}=\{\exp[\alpha(\ln|z|+iArg(z)+i2k\pi)]:k\in\mathbb{Z}\}$ \quad \qquad \qquad \qquad $\frac{d}{dz}z^\alpha=\alpha z^{\alpha-1}$ {\tiny $z\in D_{\phi}$}

\qquad\textbf{I}. If $\alpha\in\mathbb{Z}$, there is one value of $z^{\alpha}$ \quad \quad \textbf{II}. If $\alpha=\frac{p}{q},\gcd(p,q)=1,p,q\in\mathbb{Z},q\ne0$, there are exactly $q$ values of $z^{\alpha}$

\qquad\textbf{III}. If $\alpha$ is \textit{irrational} or \textit{non-real}, there are infinitely values $z^{\alpha}$ \quad \quad \textbf{IV}. $1^{1/q},q\in\mathbb{Z},q\ne0$ is $\{1,w,...,w^{q-1}\},w=\exp(i2\pi/q)$

\qquad\textbf{V}. We prefer use $\exp(z)$ to denote single-valued function, and $e^z$ to denote multi-valued function. 

\qquad\textbf{Principal Branch}: $z^{\alpha}:=\exp(\alpha Log(z))$ \qquad \qquad \textbf{Operation}: $z^{\alpha}z^{\beta}=z^{\alpha+\beta}$ {\footnotesize (Using Principal Branch)} \quad {\scriptsize \textbf{NB}: $(z_1z_2)^\alpha\ne z_1^\alpha z_2^\alpha \ ; \ (z^\alpha)^\beta\ne z^{\alpha\beta}$}


\section{Conformal Maps and Mobius Transformations} % Conformal Maps and Mobius Transformations

\textbf{Conformal}: Let $U$ be open set and $f:U\to\mathbb{C}$. Then $f$ is conformal iff: \ \ $f$ \textit{preserves angles}. \quad {\scriptsize i.e.任意两条曲线/直线之间的角度在$f$作用下不变.}

\quad \textbf{Important Theorem}: If $f:U\to\mathbb{C}$ is \textit{holomorphic}, then $\forall z_0\in U,f'(z_0)\ne0$, $f$ \textit{preserves angles}.

\quad i.e. $\forall \ \text{curves} \ C_1,C_2$ in $U$. If $C_1,C_2$ intersecting at a point $z_0\in U$. {\scriptsize $C_1,C_2$在$z_0$切线的夹角与$f(C_1),f(C_2)$在$f(z_0)$切线的夹角一样.}

\textbf{Extended Complex Plane}: $\widetilde{\mathbb{C}}:=\mathbb{C}\cup\{\infty\}$ and define that $a+\infty=\infty,b\cdot\infty=\infty,\frac{b}{0}=\infty,\frac{b}{\infty}=0$.

\textbf{Riemann Sphere}: Consider $(X,Y,Z)\in\mathbb{R}^3$: \ \ $^1z=X+iY\in\mathbb{C}$ is the point $(X,Y,0)$ and \ \ $^2 Z=0$ is the complex plane.

\begin{enumerate}[itemsep=-2pt, topsep=-2pt]
    \item Define the Riemann Sphere: $S^2:=\{(X,Y,Z)\in\mathbb{R}^3:X^2+Y^2+Z^2=1\}$ and consider the \textbf{North Pole} is point $N:=(0,0,1)$
    \item Define $\phi:\mathbb{C}\to S^2$ by \ \ $N$点 与 $z=(X,Y,0)$点 连线 与 $S^2$ 的交点为$\phi(z)$ \quad \quad Thus $\lim_{|z|\to\infty}\phi(z)=N$ \quad $\phi(\infty):=N$
    \item Calculation shows that: $\phi(z)=\phi(x+iy)=\left(\frac{2x}{|z|^2+1},\frac{2y}{|z|^2+1},\frac{|z|^2-1}{|z|^2+1}\right)$ \qquad \qquad $\psi(X,Y,Z)=\begin{cases}\frac{X+iY}{1-Z} , (X,Y,Z)\ne N \\ \infty \quad \ , \ (X,Y,Z)=N\end{cases}$ \\
    \textbf{Remark}: $\phi:\widetilde{\mathbb{C}}\to S^2$ is bijection and it's inverse $\psi:S^2\to\widetilde{\mathbb{C}}$ is the \textbf{stereographic projection}
    \item Stereographic projection $\psi(X,Y,Z)$ maps a circle to either a circle or a straight line. (见上图)
\end{enumerate}

\textbf{Mobius Transformation}: A Mobius Transformation is a function form: $f(z)=\frac{az+b}{cz+d}$ where $a,b,c,d\in\mathbb{C} \ ; \ ad\ne bc$

\begin{enumerate}[itemsep=-2pt, topsep=-2pt]
    \item \textbf{Remark}: $g(z)=\frac{f(z)}{\sqrt{ad-bc}}$ satisfies $ad-bc=1$ \quad $\big|$ \quad If $a,b,c,d$ defined a mobius transformation, then $\lambda a,\lambda b,\lambda c,\lambda d$ also.
    \item For Complex Matrix: $M=\begin{pmatrix}a & b \\ c & d\end{pmatrix}$ with $\det(M) = ad-bc=1$. \quad We define $f_{M}=\frac{az+b}{cz+d}$ \qquad $\begin{matrix}\textbf{I}. f_{M_1M_2}=f_{M_1}f_{M_2} \\ \textbf{II}. f_{M^{-1}}=f_M^{-1} \qquad \end{matrix}$
    \item Extended $f(z)$ from $\mathbb{C}$ to $\widetilde{\mathbb{C}}$ by: $f(-\frac{d}{c})=\infty$ and $f(\infty)=\frac{a}{c}$
    \item {\footnotesize \textbf{Translation}: $f(z)=z+b\Leftrightarrow \begin{pmatrix}1 & b \\ 0 & 1\end{pmatrix}$ \quad \textbf{Rotation}: $f(z)=az,a=e^{i\theta} \ (|a|=1)\Leftrightarrow \begin{pmatrix}e^{i\theta/2} & 0 \\ 0 & -e^{i\theta/2}\end{pmatrix}$ \quad \textbf{Dilation}: $f(z)=rz,r>0 \Leftrightarrow \begin{pmatrix}\sqrt{r} & 0 \\ 0 & 1/\sqrt{r}\end{pmatrix}$} \\
    {\footnotesize \textbf{Inversion}: $f(z)=1/z \Leftrightarrow \begin{pmatrix}0 & i \\ i & 0\end{pmatrix}$ \quad \textbf{$f$ fixes the point at infinity}: If $f(\infty)=\infty$ \ \ {\scriptsize ps: 除了inversion其他都是fix the point at infinity.}}
    \item {\footnotesize \textbf{Theorem}: $f(z)=\frac{az+b}{cz+d}$ be a Mobius Transformation. \ $\Rightarrow$ \ $^1$If $f(\infty)=\infty$: $f$ is a composition of \underline{finite} \textit{Translation, Rotation, Dilation} $\Rightarrow$ $c=0,f(z)=\frac{a}{d}z+\frac{b}{d}$}
    \quad {\footnotesize $^2$ If $f(\infty)<\infty$: $f$ is composition of \underline{finite} \textit{Translation, Rotation, Dilation} and \underline{only one} \textit{inversion}. $\Rightarrow$ $f(z)=\frac{(bc-ad)/c^2}{z+d/c}+\frac{a}{c}$}
\end{enumerate}

\textbf{Properties of Mobius Transformation}: \textit{Important}: $\star$ Möbius transformations map circlines to circlines. $\star$
\begin{enumerate}[itemsep=-2pt, topsep=-2pt]
    \item For mobius transformation $f(z)=\frac{az+b}{cz+d}$, if: \ $\exists z_1,z_2,z_3\in\mathbb{C}$ distinct points. $f(z_1)=z_1,f(z_2)=z_2,f(z_3)=z_3$ $\Rightarrow$ $f$ is identity.
    \item If $z_1,z_2,z_3\in\widetilde{\mathbb{C}}$ distinct points. \quad $\exists!$ mobius transformation $f(z)$ s.t. $f(z_1)=1,f(z_2)=0,f(z_3)=\infty$
    \item If $(z_1,z_2,z_3),(w_1,w_2,w_3)\in\widetilde{\mathbb{C}}$ distinct points. Then $\exists!$ mobius transformation $f(z)$ s.t. $f(z_i)=w_i \ , \ \forall i\in\{1,2,3\}$ \\
    \textbf{ps:Method to construct $2$}: If $z_i<\infty,f(z)=\frac{z_1-z_3}{z_1-z_2}\cdot\frac{z-z_2}{z-z_3}$ \qquad If $z_i=\infty$, {\footnotesize $f(z)=\frac{z-z_2}{z-z_3}$ {\scriptsize ,$z_1=\infty$} $f(z)=\frac{z_1-z_3}{z-z_3}$ {\scriptsize ,$z_2=\infty$} ; $f(z)=\frac{z-z_2}{z_1-z_2}$ {\scriptsize ,$z_3=\infty$} } \\
    \textbf{ps:Method to construct $3$}: For 3: Let $f:=h^{-1}\circ g$ {\footnotesize where $g(z_i),h(w_i)=\{1,0,\infty\}$ like part 2.}
    \vspace{1.5pt}
\end{enumerate}

\textbf{Geometric Meaning by using Mobius Transformation|Exponential|Complex Powers}:

\begin{enumerate}[itemsep=-2pt, topsep=-2pt]
    \item \textbf{Rotation}: $f(z)=e^{-i\theta}z$ is a rotation by $\theta$ (anticlockwise) about the origin. \qquad Specially, $f(z)=iz$ is a rotation by $\frac{\pi}{2}$
    \item \textbf{Extend}: $f(z)=\exp(\alpha z)$ {\small 原来的图像进行拉长,以及旋转(如果带$\theta$带$i$时)} e.g. $\{z:0<Im(z)<1\}$ {\small 可以被拉长到$\{z:0<Im(z)\}$}
    \item \textbf{Angle Extend}: $f(z)=z^{\alpha}$ {\small 原来的图像辐角范围收缩或放大}
    \item \textbf{Circlines}: \textbf{I}. 单位圆到实轴, $f(z)=\frac{z-i}{z+i}$ \quad \textbf{II}. 实轴到单位圆, $f(z)=i\frac{1+z}{1-z}$ \\
          \hspace{50pt} \textbf{III}. 单位圆到虚轴, $f(z)=\frac{z-1}{z+1}$ \quad \textbf{IV}. 虚轴到单位圆, $f(z)=\frac{1+iz}{1-iz}$
\end{enumerate} 

\textbf{Cross-Ratio}: cross-ratio $[z_1,z_2,z_3,z_4]:=f(z_1)$ where $f$ is mobius transformation s.t. $f(z_2)=1,f(z_3)=0,f(z_4)=\infty$

\begin{enumerate}[itemsep=-2pt, topsep=-2pt]
    \item \textbf{Formulas}: {\footnotesize $[z_1,z_2,z_3,z_4]=\frac{z_1-z_3}{z_1-z_4}\frac{z_2-z_4}{z_2-z_3}$ \quad $[\infty,z_2,z_3,z_4]=\frac{z_2-z_4}{z_2-z_3}$ \quad $[z_1,\infty,z_3,z_4]=\frac{z_1-z_3}{z_1-z_4}$ \quad $[z_1,z_2,\infty,z_4]=\frac{z_2-z_4}{z_1-z_4} \quad [z_1,z_2,z_3,\infty]=\frac{z_1-z_3}{z_2-z_3}$}
    \item \textbf{Theorem}: If $f$ is a mobius transformation, $[f(z_1),f(z_2),f(z_3),f(z_4)]=[z_1,z_2,z_3,z_4]$ \qquad \quad {\scriptsize \textbf{$z_i$'s in this "small section" are distinct.}}
\end{enumerate}


\section{Complex Integration} % Complex Integration

\subsection{Line Integral} % Line Integral

\textbf{Integrable}: $f:[a,b]\to \mathbb{C}$ as $f(t)=u(t)+iv(t)$ is integrable if: \ \ $u,v$ are both integrable on $[a,b]$ and for $f(t)$:

\begin{enumerate}[itemsep=-2pt, topsep=-2pt]
    \item \textbf{Def}: $\int_a^b f(t)dt:=\int_a^b u(t)dt+i\int_a^b v(t)dt$
    \item \textbf{Property I}. $\alpha f+\beta g$ is integrable and $\int_a^b(\alpha f+\beta g)dt=\alpha\int_a^b f(t)dt+\beta\int_a^b g(t)dt$
    \item \textbf{Property II}. If $f$ is \textit{continuous} and $\frac{dF}{dt}=f(t)$ for $F:[a,b]\to\mathbb{C}$ is differentiable. \ $\Rightarrow$ \ $\int_a^b f(t)dt=F(b)-F(a)$
    \item \textbf{Property III}. If $f$ is \textit{continuous} \ $\Rightarrow$ \ $\left|\int_a^b f(t)dt\right|\leq\int_a^b|f(t)|dt$.
\end{enumerate}

\textbf{Parameters Curves}: A parametrized curve connecting $z_0$ to $z_1$ is a \textit{continuous} function $\gamma:[t_0,t_1]\to\mathbb{C}$ s.t. $\gamma(t_0)=z_0,\gamma(t_1)=z_1$

\quad If {\footnotesize $z_0=x_0+iy_0,z_1=x_1+iy_1$, then $\gamma(t)=x(t)+iy(t)$ \textit{continuous} functions. s.t. $x(t_0)=x_0,x(t_1)=x_1,y(t_0)=y_0,y(t_1)=y_1$}

\quad \textbf{Regular}: $\gamma$ is regular if $\gamma'(t)\ne0$ for all $t\in[t_0,t_1]$ \qquad \textbf{Remark}: Curve $\gamma([t_0,t_1])=\Gamma$ is \textit{closed and bdd}.

\textbf{Integral Along Curve}: Let $\gamma:[t_0,t_1]\to\mathbb{C}$ be a \textit{regular} curve s.t. $\gamma([t_0,t_1])=\Gamma$ and $f:\Gamma\to\mathbb{C}$ is \textit{continuous}.

\begin{enumerate}[itemsep=-2pt, topsep=-2pt]
    \item $\star$ \textbf{Def}: $\int_{\Gamma}f(z)dz:=\int_{t_0}^{t_1}f(\gamma(t))\gamma'(t)dt$ $\star$
    \item \textbf{Circle at zero}: \textit{Circle Centred at 0 with radius R}: $\gamma:[0,1]\to\mathbb{C}$ by $\gamma(t)=R\exp(2\pi it)$
    \item \textbf{Constant Function}: If $f(z)=c$ ; $\gamma:[a,b]\to\mathbb{C}$. Then $\int_{\Gamma}f(z)dz=\int^a_b c\cdot\gamma'(z) dz = c\cdot(\gamma(b)-\gamma(a))$
\end{enumerate}

\textbf{Arclength of Curve}: {\small Let $\gamma:[t_0,t_1]\to\mathbb{C}$ be a \textit{regular} curve. $\gamma(t)=x(t)+iy(t)$ Then arclength $\ell(\Gamma):=\int_{t_0}^{t_1}|\gamma'(t)|dt=\int^{t_1}_{t_0}\sqrt{x'(t)^2+y'(t)^2}dt$}

\quad \textbf{Lemma}: If $\Gamma$ is an arc of a circle of radius $r$ traced though angle $\theta$, then $\ell(\Gamma)=r\theta$ {\small (扇形弧长)}

\textbf{Properties of Integral Along Curve}: Let $\Gamma$ be a \textit{regular} curve and $f,g:\Gamma\to\mathbb{C}$ be \textit{continuous}, and $\alpha,\beta\in\mathbb{C}$

\begin{enumerate}[itemsep=-2pt, topsep=-2pt]
    \item \textbf{M-L Lemma}: $|\int_{\Gamma}f(z)dz|\leq \max_{z\in\Gamma}|f(z)|\ell(\Gamma)$
    \item \textbf{Lemma}: $\int_{\Gamma}(\alpha f+\beta g)dz=\alpha\int_{\Gamma}f(z)dz+\beta\int_{\Gamma}g(z)dz$ \qquad $\int_{-\Gamma}f(z)dz=-\int_{\Gamma}f(z)dz$ \ \ {\scriptsize Here: $\widetilde{\gamma}(t):=\gamma(b-t)$ have $\widetilde{\gamma}([a,b])=-\Gamma$}
    \item \textbf{Change of Variables}: {\small If $^1\gamma:[a,b]\to\Gamma$, and $\widetilde{\gamma}:[\widetilde{a},\widetilde{b}]\to\Gamma$ are \textit{two parametrizations} of $\Gamma$;} \\
    {\small $^2$ $\exists\lambda:[\widetilde{a},\widetilde{b}]\to[a,b]$ s.t. $\lambda'(t)>0$ and $\widetilde{\gamma}(t)=\gamma(\lambda(t))$ \ \ {\scriptsize (防止曲线回头)} \ \ \ $\Rightarrow$ \ \ \ $\int^b_a f(\gamma(t))\gamma'(t)dt=\int^{\widetilde{b}}_{\widetilde{a}}f(\widetilde{\gamma}(t))\widetilde{\gamma}'(t)dt$}. \\
    {\small (特别的,如果$\Gamma$是closed,$f$在$\Gamma$上的积分与哪里选择起/终点无关)}
\end{enumerate}

\textbf{Contour}: A curve $\Gamma$ is \textit{contour} if it's \textit{finite union of regular curves} $\Gamma_1,\Gamma_2,...,\Gamma_n$. \quad \quad {\footnotesize Each $\Gamma_i$ is \textbf{regular component} of $\Gamma$}

\quad \textbf{Contour Integral}: If $f:\Gamma\to\mathbb{C}$ is \textit{continuous} and $\Gamma$ is a \textit{contour}. Then $\int_{\Gamma}f(z)dz:=\sum^n_{i=1}\int_{\Gamma_i}f(z)dz$


\subsection{Independent of Path} % Independent of Path

\textbf{Domain}: $D\subseteq\mathbb{C}$ is a \textit{domain} if it's \textit{open} and \textit{connected}. {\small (i.e. 任意两点都存在contour($\Gamma$)将其连接,并都在$D$里面)}

\textbf{Lemma}: Let $D\subseteq\mathbb{C}$ be a domain. If $u:D\to\mathbb{C}$ is \textit{differentiable}, with $\frac{\partial u}{\partial x}=\frac{\partial u}{\partial y}=0$. \ $\Rightarrow$ \ $u$ is \textit{constant} on $D$. \qquad \quad {\scriptsize $\Downarrow$ Clearly, $F$ is \textit{holomorphic}}

\textbf{Antiderivative}: {\small Let $D$ be a domain. For $f:D\to\mathbb{C}$ be \textit{continuous} and $F:D\to\mathbb{C}$ s.t. $F'(z)=f(z)$ for all $z\in D$. Then $F$ is an \textit{antiderivative} of $f$.}

\textbf{Fundamental Theorem of Calculus}: {\small $D$ domain; $f:D\to\mathbb{C}$ \textit{continuous}; $F:D\to\mathbb{C}$ \textit{antiderivative} of $f$. Contour $\Gamma$ in $D$ connecting $z_0$ to $z_1$.}

\hspace{160pt} Then $\int_{\Gamma}f(z)dz = F(z_1)-F(z_0)$

\begin{enumerate}[itemsep=-2pt, topsep=-2pt]
    \item $D$ domain, if $f:D\to\mathbb{C}$ is \textit{holomorphic} and $f'(z)=0,\forall z\in D$. \ $\Rightarrow$ \ $f$ is \textit{constant} on $D$.
    \item \textbf{Path-Independence Lemma}: $D$ domain, $f$ \textit{continuous} on $D$. \ Then: \\
    $f$ has \textit{antiderivative} on $D$ \ \ $\Leftrightarrow$ \ \ $\int_{\Gamma}f(z)dz=0$ $\forall$ \textit{closed contours} $\Gamma$ in $D$ \ \ $\Leftrightarrow$ \ \ $\int_{\Gamma}f(z)dz$ is \textit{path-independent}.
\end{enumerate}


\subsection{Cauchy's Theorem} % Cauchy's Theorem

\textbf{Simple}: {\small A \textit{contour} $\Gamma$ is \textit{simple} if it doesn't intersect itself except at the endpoints.} \quad \textbf{Loop}: {\small A \textit{contour} $\Gamma$ is a \textit{loop} if it's \textit{simple} and $\Gamma(t_0)=\Gamma(t_1)$}

\textbf{Jordan Curve Theorem}: {\small $\forall \ \Gamma$ be \textit{Loop} \ \ \textbf{Interior $Int(\Gamma)$}: {\footnotesize $\Gamma$ 的内部,bounded.} \ \ \textbf{Exterior $Ext(\Gamma)$}: {\footnotesize $\Gamma$ 的外部,unbounded.} \ \ \textbf{Boundary} {\footnotesize $\Gamma$ 的边界, $\Gamma$ itself.}}

\quad And $Int(\Gamma)$ is bounded \textit{domain} \quad $Ext(\Gamma)$ is unbounded \textit{domain}. \quad \textbf{Remark}: $Int(\Gamma)$ is \textit{open} and $Ext(\Gamma)$ is \textit{open} also.

$\cdot$ \textbf{Common Loop}: $C_r(z_0)$ is a circle of radius $r$ centered at $z_0$ \quad Corresponding $\gamma(t)=z_0+r\exp(2\pi i t)$ $t\in[0,1]$

$\cdot$ \textbf{Positive-Oriented}: {\small If $\Gamma$ is a \textit{loop}, then $\Gamma$ is \textit{positive-oriented} if:} \ \ {\scriptsize 按方向走时,内部在左边} {\tiny (as we move along the curve in the direction of parametrization, the interior is on the left-hand side.)}

\quad \textbf{Remark}: Unless otherwise stated, all loops shall be \textit{positively-oriented}.

\textbf{Simply-Connected}: A domain $D$ is \textit{simply-connected} if: \ \ $\forall$ \textit{loop} $\Gamma$ in $D$, $Int(\Gamma)\subseteq D$ \qquad {\footnotesize (即没有洞的domain/open set)}

\textbf{Cauchy Integral Theorem}: If $\Gamma$ is \textit{Loop}, $f$ is holomorphic in $Int(\Gamma)\cup \Gamma$ {\scriptsize (Inside and on $\Gamma$)}, then $\int_{\Gamma}f(z)dz=0$

\quad \textbf{Corollary}: If $D$ is \textit{simply-connected} domain and $f:D\to\mathbb{C}$ is \textit{holomorphic} on $D$. Then $f(z)$ has \textit{antiderivative} on $D$. $\star$ 

\qquad\qquad\quad {\scriptsize 即:在没有洞的open set上如果都是holomorphic,那么都有antiderivative.}

\quad \textbf{Remark}: {\small 如果loop $\Gamma$上和以内没有穿过任何非holomorphic点,那么$f(z)$的积分值不变.}

\textbf{Theorem}: Let $z_0\in\mathbb{C}$, $\Gamma$ be \textit{Loop}. Then $\int_\Gamma \frac{1}{z-z_0}=\begin{cases}2\pi i & if \ z_0\in\text{Int}(\Gamma) \\ 0 & \text{otherwise} \end{cases}$

\textbf{Deformation Theorem}: {\footnotesize Let $\Gamma_1,\Gamma_2$ be \textit{loops}, and $f$ is \textit{holomorphic} on $\left(Int(\Gamma_1)\setminus Int(\Gamma_2)\right)\bigcup\left(Int(\Gamma_2)\setminus Int(\Gamma_1)\right) \ , \ \Gamma_1 \ , \ \Gamma_2$. Then $\int_{\Gamma_1}f(z)dz=\int_{\Gamma_2}f(z)dz$}

\quad {\small 即: 两个\textit{loop} $\Gamma_1$ 和 $\Gamma_2$ 及它们围成的区域中(除公共区域)上,函数$f(z)$全纯, 那么它们的路径积分相等} \quad {\scriptsize ps: 可以是内外loop,也可以是交叉的loop}


\subsection{Cauchy's Integral Formula} % Cauchy's Integral Formula

\textbf{Cauchy's Integral Formula}: $\Gamma$ \textit{Loop}, $f(z)$ \textit{holomorphic} inside and on $\Gamma$, $z_0\in Int(\Gamma)$, \quad $\Rightarrow$ \quad $f(z_0)=\frac{1}{2\pi i}\int_{\Gamma}\frac{f(z)}{z-z_0}dz$

\quad {\small ps: We always use it to calculate: $\int_{\Gamma}\frac{f(z)}{z-z_0}dz$ if $f(z)$ is holomorphic on and inside $\Gamma$ (\textit{loop}), and $z_0\in Int(\Gamma)$. \quad $\Rightarrow$ \quad $\int_{\Gamma}\frac{f(z)}{z-z_0}dz=2\pi i f(z_0)$}

\textbf{Theorem}: $D$ be \textit{domain}, $\Gamma$ be \textit{contour} in $D$, $g:D\to\mathbb{C}$ \textit{continuous} on $\Gamma$, Then:

\hspace{46pt} Function Defined as: $G:D\setminus\Gamma\to\mathbb{C}$ by \quad $G(z)=\int_{\Gamma}\frac{g(w)}{w-z}dw$ is \textit{holomorphic} on $D\setminus\Gamma$ \quad and \quad $G'(z)=\int_{\Gamma}\frac{g(w)}{(w-z)^2}dw$

\hspace{46pt} Moreover, function $H:D\setminus\Gamma\to\mathbb{C}$ by \quad $H(z)=\int_{\Gamma}\frac{g(w)}{(w-z)^n}dw$ is \textit{holomorphic} on $D\setminus\Gamma$ \quad and \quad $H'(z)=n\int_{\Gamma}\frac{g(w)}{(w-z)^{n+1}}dw$

\quad $\star$   \textbf{Corollary}: {\small If $D$ is \textit{domain} and $f$ is \textit{holomorphic} on $D$, then $f$ is \textit{infinitely differentiable} on $D$, and all of its derivatives are \textit{holomorphic} on $D$.}

\textbf{Generalized Cauchy's Integral Formula}: {\footnotesize $\Gamma$ \textit{Loop}, $f(z)$ \textit{holomorphic} inside and on $\Gamma$, $z\in Int(\Gamma)$, $n\in\mathbb{N}$,} \quad $\Rightarrow$ \quad $f^{(n)}(z)=\frac{n!}{2\pi i}\int_{\Gamma}\frac{f(w)}{(w-z)^{n+1}}dw$

\quad {\small ps: We always use it to calculate: $\int_{\Gamma}\frac{f(z)}{(z-z_0)^{n+1}}dz$ {\footnotesize if $f(z)$ is holomorphic on and inside $\Gamma$ (\textit{loop}), and $z_0\in Int(\Gamma)$.} \quad $\Rightarrow$ \quad $\int_{\Gamma}\frac{f(z)}{(z-z_0)^{n+1}}dz=\frac{2\pi i}{n!}f^{(n)}(z_0)$}

\textbf{Morera Theorem}: Let $D$ is \textit{domain}, if $f:D\to\mathbb{C}$ is \textit{continuous} and $\int_{\Gamma}f(z)dz=0$ for all \textit{loop} $\Gamma$ in $D$. \ $\Rightarrow$ \ $f$ is \textit{holomorphic} on $D$.


\subsection{Liouville's Theorem, FTA and Maximum Modulus Principle} % Liouville's Theorem, FTA and Maximum Modulus Principle

\textbf{Useful Formula}: If $^1D$ domain; $^2\exists R>0,z_0\in\mathbb{C}$ s.t. $\overline{D}_{R}(z_0)\subseteq D$; $^3f$ is holomorphic on $D$

\begin{enumerate}[itemsep=-2pt, topsep=-2pt]
    \item Then $f(z_0)=\frac{1}{2\pi}\int_0^{2\pi}f(z_0+R\exp(it))dt$.
    \item If $|f(z)|<M,\forall z\in D$. Then $|f^{(n)}(z_0)|\leq\frac{n!M}{R^n}$.
    \item If $\max_{{z\in}\overline{D}_{R}(z_0)}|f(z)|=|f(z_0)|$. Then $f$ is \textit{constant} on $\overline{D}_{R}(z_0)$.
\end{enumerate}

\textbf{Criteria Constant Function}: If $f:\mathbb{C}(or \ D)\to\mathbb{C}$ is \textit{holomorphic} and \textit{bounded} on: \qquad\qquad $D$ domain

\begin{enumerate}[itemsep=-2pt, topsep=-2pt]
    \item \textbf{Liouville's Theorem}: $|f(z)|<M$ \textit{bounded} on $\forall z \in \mathbb{C}$, \ $\Rightarrow$ \ $f(z)$ is \textit{constant}.
    \item \textbf{Maximum Modulus Principle}: $|f(z)|$ \textit{bounded} on $\forall z \in D$, and $|f(z)|$ has \textit{maximum} at $z_0\in D$. \ $\Rightarrow$ \ $f(z)$ is \textit{constant}.
    \textbf{Remark I}: {\footnotesize 意思是对于$f(z)$ holomorphic且在domain上bounded,如果 $|f(z)|$ 在domain上有最大值(非边界),那么$f(z)$是constant.} \\
    \textbf{Remark II} : $\star$ If function $f$ is \textit{holomorphic} on a \textit{bounded domain} $D$ and \textit{continuous} up to the boundary of $D$. \\
    $\Rightarrow$ \ $f$ has \textit{maximum} modulus on the boundary of $D$. \qquad {\tiny 若\( f \) 在 \( D \) 内全纯,且在 \( \partial D \) 上连续,则 \( f \) 在$D\cup\partial D$最大值一定在边界上.特别地,若 \( f \) 不是常数,则最大值只能在边界上取到.}
    \item {\small \textbf{Maximum/Minimum Principle for Harmonic Functions}: If $D$ domain, $\phi:D\to\mathbb{R}$ is \textit{harmonic}, and $\phi$ is \textit{bounded above/below} on $D$ by $M$,} \\
    {\small with $\phi(z_0)=M$ for some $z_0\in D$.} \ $\Rightarrow$ \ $\phi$ is \textit{constant} on $D$. \\
    \textbf{Remark}: {\footnotesize 对于调和函数$\phi:D\to\mathbb{R}$,如果$f$不是常数,那么最大值只能在边界上取到.}
\end{enumerate}

\textbf{Fundamental Theorem of Algebra}: If $P:\mathbb{C}\to\mathbb{C}$ is a non-constant \textit{polynomial}. \ $\Rightarrow$ \ $P$ has a at least one \textit{root} in $\mathbb{C}$.


\section{infinity Series} % Infinity Series

\subsection{Basic Properties, Convergence Test, Series of Functions and M-Test} % Basic Properties, Convergence Test, Series of Functions and M-Test

\textbf{Partial Sum}: A Series $\sum_{n=0}^{\infty}z_n$ is \textit{convergent} if \textit{partial sums} $S_n = \sum_{k=0}^{n}z_k$ is \textit{convergent}. \quad \textbf{Remark}: $\sum z_n$ is \textit{convergent} $\Rightarrow$ $\lim z_n=0$.

\textbf{Comparison Test}: If $|z_n|\leq M_n$ for all $n\in\mathbb{N}$ and $\sum M_n$ is \textit{convergent}. \ $\Rightarrow$ \ $\sum z_n$ is \textit{convergent}.

\textbf{Lemma|`Geometric Series'}: For $c\in\mathbb{C}$, $\sum_{n=0}^\infty c^n$ is \textit{convergent} $\Leftrightarrow$ $|c|<1$. \quad \textbf{Remark}: $\sum_{n=0}^\infty c^n=\frac{1}{1-c}$

\textbf{Ratio Test}: {\small For $\sum z_n$, let $L = \lim_{n\to\infty}\left|\frac{z_{n+1}}{z_n}\right|$. \quad If $L<1$, then $\sum z_n$ is \textit{convergent}. \quad If $L>1$, then $\sum z_n$ is \textit{divergent}.} \quad {\footnotesize If $L=1$, conclude nothing.}

\textbf{Converge Pointwise}: {\small Seq $f_n: S\to\mathbb{C}$ \textit{pointwise convergent} to $f:S\to\mathbb{C}$ if \quad $\forall \varepsilon>0, \forall z \in S, \exists N_{\varepsilon,z}\in\mathbb{N}$ s.t. $|f_n(z)-f(z)|<\varepsilon$ for all $n\geq N$}

\textbf{Uniform Convergence}: {\small Seq $f_n: S\to\mathbb{C}$ \textit{uniformly convergent} to $f:S\to\mathbb{C}$ if \quad $\forall \varepsilon>0, \exists N_{\varepsilon}\in\mathbb{N}$ s.t. $|f_n(z)-f(z)|<\varepsilon$ for all $n\geq N$ and $\forall z\in S$}

\quad \textbf{Lemma}: If $f_n:S\to\mathbb{C}$ is \textit{uniformly convergent} and \textit{continuous} to $f:S\to\mathbb{C}$, then $f$ is \textit{continuous} on $S$.\quad {\footnotesize (i.e. Uniform 保留 连续性)}

\textbf{Weierstrass M-Test}: 


\section{Appendix}

\subsection{Convergence Test for Real Series} % Convergence Test for Real Series

\textbf{Divergence Test}: If $\lim a_n\ne 0 \Rightarrow \sum a_n$ diverges. \ \ (If $\sum a_n$ convergent $\Rightarrow \lim a_n=0.$) \qquad \textbf{p-Test}: $\sum \frac{1}{n^p}$ convergent iff $p>1$

\textbf{Comparison Test}: If $0<a_n<b_n$, $\sum b_n$ convergent $\Rightarrow \sum a_n$ also ; $\sum a_n$ divergent $\Rightarrow \sum b_n$ also.

\textbf{Integral Test}: Let $f:[1,\infty)\rightarrow\mathbb{R}$ is 非负递减, $a_n=f(n)$. Then $\sum a_n$ converges iff $\int_{1}^{\infty}f(x)dx<\infty.$

\textbf{Absolutely Convergence}: $\sum a_n$ convergent absolutely iff $\sum |a_n|$ convergent. \ \ \textbf{If convergent abs $\Rightarrow$ convergent.}

\textbf{Alternating Series Test}: If $a_n$ decreasing, $a_n\geq 0,\lim a_n = 0$. Then $\sum (-1)^{n-1}a_n$ convergent.

\textbf{Cauchy's Condensation Test}: If $a_n\geq 0,a_n$ decreasing, $\Rightarrow \ [\ \sum a_n convergent \Leftrightarrow \sum 2^na_{2^n} \ also \ ]$




\end{document}